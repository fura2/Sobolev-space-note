\documentclass[a4paper,dvipdfmx]{jarticle}

\usepackage{amscd}
\usepackage{amsmath}
\usepackage{amssymb}
\usepackage{amsthm}
\usepackage{cite}
\usepackage{enumitem}
\usepackage{fnpct}
\usepackage[top=30truemm,bottom=30truemm,left=20truemm,right=20truemm]{geometry}
\usepackage{hhline}
\usepackage[colorlinks=true,allcolors=blue]{hyperref}
\usepackage{mathtools}
\usepackage[refpage]{nomencl}
\usepackage{pxjahyper}

\numberwithin{equation}{section}

\renewcommand{\labelenumi}{\roman{enumi})}

\newcommand{\Cbb}{\mathbb{C}}
\newcommand{\Qbb}{\mathbb{Q}}
\newcommand{\Rbb}{\mathbb{R}}
\newcommand{\Tbb}{\mathbb{T}}
\newcommand{\Zbb}{\mathbb{Z}}

\newcommand{\crm}{\mathrm{c}}
\newcommand{\drm}{\mathrm{d}}
\newcommand{\Lrm}{\mathrm{L}}
\newcommand{\Trm}{\mathrm{T}}

\newcommand{\Ocal}{\mathcal{O}}

\newcommand{\Afrak}{\mathfrak{A}}
\newcommand{\Bfrak}{\mathfrak{B}}
\newcommand{\Ffrak}{\mathfrak{F}}
\newcommand{\Mfrak}{\mathfrak{M}}
\newcommand{\cfrak}{\mathfrak{c}}

\newcommand{\e}{\mathrm{e}}
\newcommand{\im}{\mathrm{i}}

\renewcommand{\epsilon}{\varepsilon}
\renewcommand{\Re}{\operatorname{Re}}
\renewcommand{\Im}{\operatorname{Im}}

\DeclareMathOperator{\trace}{tr}
\DeclareMathOperator{\gKer}{gKer}
\DeclareMathOperator{\Ker}{Ker}
\DeclareMathOperator{\Ran}{Ran}
\DeclareMathOperator{\lspan}{span}
\DeclareMathOperator{\rank}{rank}
\DeclareMathOperator{\sign}{sign}
\DeclareMathOperator{\sech}{sech}
\DeclareMathOperator{\csch}{csch}
\DeclareMathOperator{\sinc}{sinc}
\DeclareMathOperator*{\esssup}{ess\,sup}

\newcommand{\defeq}{\mathrel{\rlap{\raisebox{0.3ex}{$\cdot$}}\raisebox{-0.3ex}{$\cdot$}}=}

% https://tex.stackexchange.com/questions/31354/how-to-remove-the-in-the-proof-environment
\makeatletter
\renewenvironment{proof}[1][\proofname]{\par
  \pushQED{\qed}%
  \normalfont \topsep6\p@\@plus6\p@\relax
  \trivlist
  \item[\hskip\labelsep
%        \itshape
        \bfseries
%    #1\@addpunct{.}]\ignorespaces
    {#1}]\ignorespaces
}{%
  \popQED\endtrivlist\@endpefalse
}
\makeatother

\theoremstyle{definition}
\newtheorem{definition}{定義}[section]
\newtheorem{lemma}[definition]{補題}
\newtheorem{proposition}[definition]{命題}
\newtheorem{theorem}[definition]{定理}
\newtheorem{corollary}[definition]{系}
\newtheorem{remark}[definition]{注意}
\newtheorem{example}[definition]{例}
\newtheorem{claim}{主張}
\renewcommand\proofname{証明}

\makenomenclature
\renewcommand{\nomname}{記号索引}
\renewcommand*\pagedeclaration[1]{\qquad #1}

\begin{document}

\section{Lebesgue 積分}

後に必要になる内容に焦点を当てて Lebesgue 積分論を眺める.
この節では主に \cite{It63,mpmi} を参考にした.

\subsection{Lebesgue測度の構成}

\begin{definition}
    $X$ を集合とする.
    次の $3$ つの条件を満たす $\Afrak\subset2^X$ を $X$ 上の $\sigma$-加法族 ($\sigma$-algebra) という.
    \begin{enumerate}[label=\textsf{(SA\arabic*)},align=left]
        \item\label{item:sa1} $\emptyset\in\Afrak$.
        \item\label{item:sa2} 任意の $A\in\Afrak$ に対して $A^\crm\in\Afrak$.\quad (ここで,$A^\crm\defeq X\setminus A$)
        \item\label{item:sa3} 任意の $(A_i)_{i=1}^\infty\subset\Afrak$ に対して $\displaystyle\bigcup_{i=1}^\infty A_i\in\Afrak$.
    \end{enumerate}
    組 $(X,\Afrak)$ を可測空間 (measurable space),$\Afrak$ の元を可測集合 (measurable set) という.
\end{definition}

\begin{remark}
    可測空間 $(X,\Afrak)$ に対して次が成り立つ.
    \begin{enumerate}
        \item $X\in\Afrak$.
        \item 任意の $(A_i)_{i=1}^\infty\subset\Afrak$ に対して $\displaystyle\bigcap_{i=1}^\infty A_i\in\Afrak$.
    \end{enumerate}
    前者は \ref{item:sa1} と \ref{item:sa2} から,後者は \ref{item:sa2} と \ref{item:sa3} から従う.
\end{remark}

\begin{example}
    $\Afrak=\{\emptyset,X\},2^X$ はいずれも $X$ 上の $\sigma$-加法族になる.
\end{example}

\begin{example}
    $X$ 上の $\sigma$-加法族の族 $\{\Afrak_\lambda\}_{\lambda\in\Lambda}$ に対して,
        $\bigcap_{\lambda\in\Lambda}\Afrak_\lambda$
    も $X$ 上の $\sigma$-加法族になる.

    $S\subset 2^X$に対して,$S$ を含む (包含関係について) 最小の $\sigma$-加法族が
    \begin{align*}
        \bigcap_{\substack{\text{$\Afrak$: $\sigma$-algebra on $X$}\\S\subset\Afrak}}\Afrak
    \end{align*}
    として一意に定まる.
    これを $S$ が生成する $\sigma$-加法族という.
\end{example}

\begin{example}
    位相空間 $(X,\Ocal)$ に対して,$\Ocal$ が生成する $X$ 上の $\sigma$-加法族を Borel 集合族 (Borel algebra) といい,$\Bfrak(X)$ と書く.
    $\Bfrak(X)$ の元を Borel 集合という.
    また,$\Bfrak^n\defeq\Bfrak(\Rbb^n)$ と置く.
    \nomenclature{$\Bfrak(X)$}{位相空間 $X$ 上の Borel 集合族}
    \nomenclature{$\Bfrak^n$}{$\Rbb^n$ 上の Borel 集合族}
\end{example}

\begin{example}
    実数体 $\Rbb$ に形式的な元 $+\infty,-\infty$ を加えた集合 $\Rbb\cup\{\pm\infty\}$ を考える.
    $a\in\Rbb$ に対して,順序関係を $-\infty<a<+\infty$,四則演算を
    \begin{align*}
        &(\pm\infty)+a=a+(\pm\infty)\defeq\pm\infty,\quad
        (\pm\infty)+(\pm\infty)\defeq\pm\infty,\\
        &(\pm\infty)-a=a-(\pm\infty)\defeq\mp\infty,\quad
        (\pm\infty)-(\mp\infty)\defeq\pm\infty,\\
        &a\cdot(\pm\infty)=(\pm\infty)\cdot a\defeq\begin{cases}
            \pm\infty&\text{($0<a<+\infty$ のとき)}\\
            \mp\infty&\text{($-\infty<a<0$ のとき)},
        \end{cases}\\
        &(\pm\infty)\cdot(\pm\infty)\defeq+\infty,\quad
        (\pm\infty)\cdot(\mp\infty)\defeq-\infty,\\
        &a/(\pm\infty)\defeq0,\\
        &(\pm\infty)/a\defeq\begin{cases}
            \pm\infty&\text{($0<a<+\infty$ のとき)}\\
            \mp\infty&\text{($-\infty<a<0$ のとき)}
        \end{cases}
    \end{align*}
    (すべて複号同順) と定める.
    これら以外の四則演算は定義しない\footnote{測度論では $0\cdot(\pm\infty)=(\pm\infty)\cdot0\defeq0$ と定めることがあるが,\cite[p.12]{It63} に倣ってこの約束は用いずに進める.}.

    $\Ocal$ を $\Rbb$ の通常の位相とする.
    $\Rbb\cup\{\pm\infty\}$ を,位相
    \begin{align*}
        \overline\Ocal\defeq\Ocal\cup\{O\cup(a,+\infty]\mid O\in\Ocal, a\in\Rbb\}\cup\{O\cup[-\infty,a)\mid O\in\Ocal, a\in\Rbb\}\cup\{\Rbb\cup\{\pm\infty\}\}
    \end{align*}
    を備えた位相空間と見なす.
    % \begin{align*}
    %     \Ocal\cup\{(a,+\infty]\mid a\in\Rbb\}\cup\{[-\infty,a)\mid a\in\Rbb\}
    % \end{align*}
    % が生成する位相 $\overline\Ocal$ を入れる.
    このとき,$\Rbb\cup\{\pm\infty\}$ は閉区間 $[0,1]$ と同相になる.
    また,$\Ocal=\{O\cap\Rbb\mid O\in\overline\Ocal\}$ が成り立つ.
    すなわち,$\Rbb\cup\{\pm\infty\}$ の部分位相空間としての $\Rbb$ は $(\Rbb,\Ocal)$ と等しい.

    $\Rbb\cup\{\pm\infty\}$ 上の Borel 集合族 $\Bfrak(\Rbb\cup\{\pm\infty\})$
    !!!TODO!!!
\end{example}

\begin{definition}
    $(X,\Afrak)$ を可測空間とする.
    次の $2$ つの条件を満たす $\mu:\Afrak\to\Rbb\cup\{+\infty\}$ を $X$ 上の測度 (measure) という.
    \begin{enumerate}[label=\textsf{(M\arabic*)},align=left]
        \item\label{item:m1} 任意の $A\in\Afrak$ に対して $\mu(A)\ge0$.また,$\mu(\emptyset)=0$.\qquad (非負性)
        \item\label{item:m2} 互いに素\footnote{pairwise disjoint.すなわち,$i\ne j$ ならば $A_i\cap A_j=\emptyset$.}な任意の $(A_i)_{i=1}^\infty\subset\Afrak$ に対して
            $\displaystyle\mu\biggl(\bigcup_{i=1}^\infty A_i\biggr)=\sum_{i=1}^\infty\mu(A_i)$.\qquad ($\sigma$-加法性)
    \end{enumerate}
    三つ組 $(X,\Afrak,\mu)$ を測度空間 (measure space) という.
\end{definition}

\begin{remark}
    測度空間 $(X,\Afrak,\mu)$ に対して次が成り立つ.
    \begin{enumerate}
        \item 任意の $A,B\in\Afrak,\ A\subset B,$ に対して $\mu(A)\le\mu(B)$.\qquad(単調性)
        \item 任意の $(A_i)_{i=1}^\infty\subset\Afrak$ に対して $\displaystyle\mu\biggl(\bigcup_{i=1}^\infty A_i\biggr)\le\sum_{i=1}^\infty\mu(A_i)$.\qquad(劣加法性)
    \end{enumerate}
\end{remark}

測度を具体的に構成するためには,次に示す外測度の概念が有用である.

\begin{definition}
    $X$ を集合とする.
    次の $3$ つの条件を満たす $\Gamma:2^X\to\Rbb\cup\{+\infty\}$ を $X$ 上の外測度 (outer measure) という.
    \begin{enumerate}[label=\textsf{(OM\arabic*)},align=left]
        \item\label{item:om1} 任意の $A\subset X$ に対して $\Gamma(A)\ge0$.また,$\Gamma(\emptyset)=0$.
        \item\label{item:om2} 任意の $A,B\subset X,\ A\subset B,$ に対して $\Gamma(A)\le\Gamma(B)$.
        \item\label{item:om3} 任意の $(A_i)_{i=1}^\infty\subset\Afrak$ に対して $\displaystyle\Gamma\biggl(\bigcup_{i=1}^\infty A_i\biggr)\le\sum_{i=1}^\infty\Gamma(A_i)$.
    \end{enumerate}
\end{definition}

\begin{example}\label{ex:Lebesgue_outer_measure}
    $I=(a_1,b_1]\times\cdots\times(a_n,b_n],\ -\infty\le a_i<b_i\le +\infty,$ と表される集合を $\Rbb^n$ の区間という.
    ただし,$b_i=+\infty$ のときは $(a_i,b_i]\defeq(a_i,+\infty)$ と約束する.
    また,
    \begin{align*}
        \mathrm{vol}(I)\defeq(b_1-a_1)\cdots(b_n-a_n)\in\Rbb\cup\{+\infty\}
    \end{align*}
    と置く.
    $\mu_\Lrm^{n,\ast}:2^{\Rbb^n}\to\Rbb\cup\{+\infty\}$ を任意の $A\subset X$ に対して
    \begin{align*}
        \mu_\Lrm^{n,\ast}(A)\defeq\inf\left\{
            \sum_{i=1}^\infty\mathrm{vol}(I_i)
            \,\middle\vert\,
            \text{$I_i$ : $\Rbb^n$ の区間,$A\subset\bigcup_{i=1}^\infty I_i$}
        \right\}
    \end{align*}
    と定めると,これは $\Rbb^n$ 上の外測度になる.
    % 直感的には,$A$ の``大きさ''を $A$ を区間の和集合で覆って近似することによって測っている.
    $\mu_\Lrm^{n,\ast}$ を $n$ 次元 Lebesgue 外測度という.
    \nomenclature{$\mu_\Lrm^{n,\ast}$}{$n$ 次元 Lebesgue 外測度}
\end{example}

\begin{definition}
    $\Gamma$ を集合 $X$ 上の外測度,$E\subset X$ とする.
    $E$ が $\Gamma$-可測 ($\Gamma$-measurable) であるとは,Carath\'eodory の条件:任意の $A\subset X$ に対して
    \begin{align}
        \Gamma(A)=\Gamma(A\cap E)+\Gamma(A\cap E^\crm),
        \label{eqn:Caratheodory}
    \end{align}
    が成り立つことをいう.
    $\Gamma$-可測集合の全体を $\Mfrak_\Gamma$ とおく.
    \nomenclature{$\Mfrak_\Gamma$}{$\Gamma$-可測集合の全体}
\end{definition}

\begin{remark}
    \ref{item:om3} によって,\eqref{eqn:Caratheodory} の $(\text{左辺})\le(\text{右辺})$ はいつも成り立つ.
\end{remark}

\begin{theorem}\label{thm:outer_measure_to_measure}
    $\Gamma$ を集合 $X$ 上の外測度とする.
    このとき,$\Mfrak_\Gamma$ は $X$ 上の $\sigma$-加法族で
    \begin{align*}
        \Gamma\vert_{\Mfrak_\Gamma}:\Mfrak_\Gamma\to\Rbb\cup\{+\infty\}
    \end{align*}
    は $X$ 上の測度になる.
\end{theorem}

\begin{remark}
    $\Gamma=\mu_\Lrm^{n,\ast}$ の場合に \eqref{eqn:Caratheodory} が何を意味しているかを調べる.
    $E\subset A$ とし,$A$ を $E$ を含む部分 ($A\cap E$) と $E$ を含まない部分 ($A\cap E^\crm$) に分けることを考える.
    さらに,$A$ は有界な区間とする.

    このとき,\eqref{eqn:Caratheodory} を次のように書き換える.
    \begin{align*}
        \mu_\Lrm^{n,\ast}(A)=\mu_\Lrm^{n,\ast}(A\cap E)+\mu_\Lrm^{n,\ast}(A\cap E^\crm)
        &\iff\mathrm{vol}(A)=\mu_\Lrm^{n,\ast}(E)+\mu_\Lrm^{n,\ast}(A\setminus E)\\
        &\iff\mu_\Lrm^{n,\ast}(E)=\mathrm{vol}(A)-\mu_\Lrm^{n,\ast}(A\setminus E).
    \end{align*}
    最後の式の右辺は,$E$ の``大きさ''を $E$ に含まれる区間の和集合で近似することによって測ることに相当している.
    これは次のようにして分かる.
    \begin{align*}
        \mathrm{vol}(A)-\mu_\Lrm^{n,\ast}(A\setminus E)
        &=\mathrm{vol}(A)-\inf\left\{
            \sum_{i=1}^\infty\mathrm{vol}(I_i)
            \,\middle\vert\,
            \text{$I_i$ : $\Rbb^n$ の区間,$A\setminus E\subset\bigcup_{i=1}^\infty I_i$}
        \right\}\\
        &=\mathrm{vol}(A)-\inf\left\{
            \sum_{i=1}^\infty\mathrm{vol}(I_i)
            \,\middle\vert\,
            \text{$I_i$ : $\Rbb^n$ の互いに素な区間,$A\setminus E\subset\bigcup_{i=1}^\infty I_i\subset A$}
        \right\}\\
        &=\sup\left\{
            \mathrm{vol}(A)-\sum_{i=1}^\infty\mathrm{vol}(I_i)
            \,\middle\vert\,
            \text{$I_i$ : $\Rbb^n$ の互いに素な区間,$A\setminus E\subset\bigcup_{i=1}^\infty I_i\subset A$}
        \right\}\\
        &=\sup\left\{
            \sum_{j=1}^\infty\mathrm{vol}(J_j)
            \,\middle\vert\,
            \text{$J_j$ : $\Rbb^n$ の互いに素な区間,$\bigcup_{j=1}^\infty J_j\subset E$}
        \right\}.
    \end{align*}
    最後の等号では,$A$ を互いに素な区間 $I_1,I_2,\ldots,J_1,J_2,\ldots$ に分割したときに
    \begin{itemize}
        \item $\displaystyle\mathrm{vol}(A)=\sum_{i=1}^\infty\mathrm{vol}(I_i)+\sum_{j=1}^\infty\mathrm{vol}(J_j)$
        \item $\displaystyle A\setminus E\subset\bigcup_{i=1}^\infty I_i\subset A\stackrel{\text{$A\setminus\cdot$ をとる}}{\iff}\bigcup_{j=1}^\infty J_j\subset E$
    \end{itemize}
    となることを用いた.

    よって,$\mu_{\Lrm,\ast}^n(E)\defeq\mathrm{vol}(A)-\mu_\Lrm^{n,\ast}(A\setminus E)$ (``Lebesgue 内測度'') と置くと,Carath\'eodory の条件は
    \begin{align*}
        \mu_\Lrm^{n,\ast}(E)=\mu_{\Lrm,\ast}^n(E)
    \end{align*}
    と書ける (Riemann積分論における Jordan 可測性と類似).
    Lebesgue 積分論では,測度の構成のために外測度だけで事足りるので,内測度はあまり使われないようである.
\end{remark}

\begin{proof}[定理 \ref{thm:outer_measure_to_measure} の証明]
    \ref{item:sa1}, \ref{item:sa2}, \ref{item:m1} は明らか.
    \ref{item:sa3}, \ref{item:m2} を示す.
    互いに素な $(E_i)_{i=1}^\infty\subset\Mfrak_\Gamma$ を任意にとり,$S=\bigcup_{i=1}^\infty E_i$ と置く.

    \begin{claim}\label{claim:out1}
        任意の $A\subset X$ と $m\in\Zbb_{>0}$ に対して
        \begin{align}
            \Gamma(A)\ge\sum_{i=1}^m\Gamma(A\cap E_i)+\Gamma(A\cup S^\crm).
            \label{eqn:outer_measure_claim1}
        \end{align}
    \end{claim}

    \begin{proof}[主張 \ref{claim:out1} の証明]
        $m$ について帰納法で示す.
        $m=1$ のとき,
        \begin{align*}
            \Gamma(A)
            &=\Gamma(A\cap E_1)+\Gamma(A\cap E_1^\crm)&&\because E_1\in\Mfrak_\Gamma\\
            &\ge\Gamma(A\cap E_1)+\Gamma(A\cap S^\crm)&&\because\text{\ref{item:om2}}
        \end{align*}
        となるので成立.
        $m$ で成立を仮定する.
        \eqref{eqn:outer_measure_claim1} の $A$ として $A\cap E_{m+1}^\crm$ を選ぶと
        \begin{align*}
            \Gamma(A\cap E_{m+1}^\crm)
            &\ge\sum_{i=1}^m\Gamma(A\cap E_{m+1}^\crm\cap E_i)+\Gamma(A\cap E_{m+1}^\crm\cap S^\crm)\\
            &=\sum_{i=1}^m\Gamma(A\cap E_i)+\Gamma(A\cap S^\crm).
        \end{align*}
        ゆえに,
        \begin{align*}
            \Gamma(A)
            &=\Gamma(A\cap E_{m+1})+\Gamma(A\cap E_{m+1}^\crm)&&\because E_{m+1}\in\Mfrak_\Gamma\\
            &\ge\Gamma(A\cap E_{m+1})+\sum_{i=1}^m\Gamma(A\cap E_i)+\Gamma(A\cap S^\crm)\\
            &=\sum_{i=1}^{m+1}\Gamma(A\cap E_i)+\Gamma(A\cap S^\crm)
        \end{align*}
        となって $m+1$ でも成立.
    \end{proof}

    よって,任意の $A\subset X$ に対して
    \begin{align*}
        \Gamma(A)
        &\ge\sum_{i=1}^\infty\Gamma(A\cap E_i)+\Gamma(A\cap S^\crm)&&\because\text{主張 \ref{claim:out1} で $m\to+\infty$}\\
        &\ge\Gamma\biggl(\bigcup_{i=1}^\infty(A\cap E_i)\biggr)+\Gamma(A\cap S^\crm)&&\because\text{\ref{item:om3}}\\
        &=\Gamma(A\cap S)+\Gamma(A\cap S^\crm)\\
        &\ge\Gamma(A)&&\because\text{\ref{item:om3}}.
    \end{align*}
    ゆえに,
    \begin{align*}
        \Gamma(A)
        =\sum_{i=1}^\infty\Gamma(A\cap E_i)+\Gamma(A\cap S^\crm)
        =\Gamma(A\cap S)+\Gamma(A\cap S^\crm)
    \end{align*}
    となり $S\in\Mfrak_\Gamma$ が分かる.
    $A=S$ として $\Gamma(S)=\sum_{i=1}^\infty\Gamma(E_i)$ もいえるので,\ref{item:m2} が成立.

    \begin{claim}\label{claim:out2}
        任意の $E,F\in\Mfrak_\Gamma$ に対して $E\cap F,E\setminus F\in\Mfrak_\Gamma$.
    \end{claim}
    \begin{proof}[主張 \ref{claim:out2} の証明]
        \begin{align*}
            &\Gamma(A\cap E\cap F)+\Gamma(A\cap(E\cap F)^\crm)\\
            &=\Gamma(A\cap E\cap F)+\Gamma(A\cap(E\cap F)^\crm\cap E)+\Gamma(A\cap(E\cap F)^\crm\cap E^\crm)&&\because E\in\Mfrak_\Gamma\\
            &=\Gamma(A\cap E\cap F)+\Gamma(A\cap E\cap F^\crm)+\Gamma(A\cap E^\crm)\\
            &=\Gamma(A\cap E)+\Gamma(A\cap E^\crm)&&\because F\in\Mfrak_\Gamma\\
            &=\Gamma(A)&&\because E\in\Mfrak_\Gamma
        \end{align*}
        となるので $E\cap F\in\Mfrak_\Gamma$.
        また,$E\setminus F=E\cap F^\crm\in\Mfrak_\Gamma$.
    \end{proof}

    互いに素とは限らない $(E_i)_{i=1}^\infty\subset\Mfrak_\Gamma$ に対して,主張 \ref{claim:out2} より
    \begin{align*}
        \bigcup_{i=1}^\infty E_i=E_1\cup(E_2\setminus E_1)\cup(E_3\setminus(E_1\cup E_2))\cup\cdots\in\Mfrak_\Gamma
    \end{align*}
    となって,\ref{item:sa3} が成立.
\end{proof}

\begin{example}
    $X=\Rbb^n,\Gamma=\mu_\Lrm^{n,\ast}$ (例 \ref{ex:Lebesgue_outer_measure} を参照) に対して,
    定理 $\ref{thm:outer_measure_to_measure}$ の方法で定まる $\sigma$-加法族を $\Bfrak_\Lrm^n$,測度を $\mu_\Lrm^n:\Bfrak_\Lrm^n\to\Rbb\cup\{+\infty\}$ と書く.
    $\Bfrak_\Lrm^n$ の元を Lebesgue 可測集合,$\mu_\Lrm^n$ を $n$ 次元 Lebesgue 測度という.
    \nomenclature{$\Bfrak_\Lrm^n$}{Legesgue 可測集合の全体}
    \nomenclature{$\mu_\Lrm^n$}{$n$ 次元 Lebesgue 測度}
\end{example}

\begin{definition}
    $(X,\Afrak,\mu)$ を測度空間とする.
    \begin{enumerate}
        \item $N\in\Afrak$ が零集合 (null set) であるとは,$\mu(N)=0$ となることをいう.
        \item $(X,\Afrak,\mu)$ が完備 (complete) であるとは,零集合の任意の部分集合が $\Afrak$ に属する (したがって零集合になる) ことをいう.
    \end{enumerate}
\end{definition}

\begin{proposition}
    定理 \ref{thm:outer_measure_to_measure} の測度空間 $(X,\Mfrak_\Gamma,\Gamma\vert_{\Mfrak_\Gamma})$ は完備.
    特に,$(\Rbb^n,\Bfrak_\Lrm^n,\mu_\Lrm^n)$ は完備.
\end{proposition}

\begin{proof}
    零集合 $N\in\Afrak$ と任意の $N'\subset N$ に対して,$\Gamma(N')\le\Gamma(N)=0$ なので $\Gamma(N')=0$.
    よって,任意の $A\subset X$ に対して
    \begin{align*}
        \Gamma(A)
        \le\Gamma(A\cap N')+\Gamma(A\cap N'^\crm)
        =\Gamma(A\cap N'^\crm)
        \le\Gamma(A)
    \end{align*}
    となり $N'\in\Mfrak_\Gamma$ を得る.
\end{proof}

\begin{proposition}\label{prop:measure_completion}
    測度空間 $(X,\Afrak,\mu)$ に対して,$\mu^\ast:2^X\to\Rbb\cup\{+\infty\}$ を
    \begin{align*}
        \mu^\ast(A)\defeq\inf\left\{
            \mu(B)\,\middle\vert\,A\subset B\in\Afrak
        \right\}
    \end{align*}
    と定めると,これは $X$ 上の外測度になる.
    $\mu^\ast$ に対して定理 \ref{thm:outer_measure_to_measure} の方法で定まる完備測度空間
    $(X,\Mfrak_{\mu^*},\mu^\ast\vert_{\Mfrak_{\mu^\ast}})$ は
    \begin{gather*}
        \Afrak\subset\Mfrak_{\mu^*},\quad
        \mu^\ast\vert_\Afrak=\mu
    \end{gather*}
    を満たす.
\end{proposition}

\begin{definition}
    命題 \ref{prop:measure_completion} の測度空間 $(X,\Mfrak_{\mu^*},\mu^\ast\vert_{\Mfrak_{\mu^\ast}})$ を
    $(X,\Afrak,\mu)$ の完備化 (completion) という.
\end{definition}

常に完備化したほうがいいというわけではない!
$\to$ 可測関数の議論(次回)を参照

\begin{remark}
    \leavevmode
    \begin{enumerate}
        \item
            $\Rbb^n$ の Borel 集合は Lebesgue 可測である.
            すなわち $\Bfrak^n\subset\Bfrak_\Lrm^n$.
            また,測度空間 $(\Rbb^n,\Bfrak^n,\mu_\Lrm^n\vert_{\Bfrak^n})$ の完備化は
            $(\Rbb^n,\Bfrak_\Lrm^n,\mu_\Lrm^n)$ と一致する.
            前者については \cite[定理 7.2]{It63} を,後者については \cite[p.49]{It63} を参照.

        \item
            $\Bfrak^n\subsetneq\Bfrak_\Lrm^n\subsetneq2^{\Rbb^n}$ である.
            より正確には,$\cfrak$ を連続体濃度として
            \begin{align}
                |\Bfrak^n|=\cfrak,\quad
                |\Bfrak_\Lrm^n|=2^\cfrak,\quad
                |2^{\Rbb^n}|=2^\cfrak,\quad
                |2^{\Rbb^n}\setminus\Bfrak_\Lrm^n|=2^\cfrak
            \end{align}
            が成り立つ.

            一つ目の等号は超限帰納法を用いて Borel 集合を具体的に``生成''することによって示される\cite{ms70880}.
            二つ目の等号は次のようにして示される:
            $N\in\Bfrak_\Lrm^n$ で $\mu_\Lrm^n(N)=0$ かつ $|N|=\cfrak$ となるもの (たとえば Cantor 集合) を一つとる.
            Lebesgue 測度の完備性から $N$ の任意の部分集合は Lebesgue 可測なので結論が従う.
            三つ目の等号は $|\Rbb^n|=\cfrak$ から従う.
            四つ目の等号は,$A\in2^{\Rbb^n}\setminus\Bfrak_\Lrm^n$ を固定したとき,
            任意の $X\in\Bfrak_\Lrm^n$ に対して $A\ominus X\in\Bfrak_\Lrm^n$ となることから従う ($\ominus$ は集合の対称差).

        \item Lebesgue 非可測集合の例
            \begin{enumerate}
                \item
                    区間 $G=[0,1)$ は $\mathrm{mod}\ 1$ での和について Abel 群をなす.
                    $G$ を部分群 $H=\Qbb\cap[0,1)$ で割った剰余群 $G/H$ について,その完全代表系 $A$ を一つとり固定する.
                    このとき,$G=\bigcup_{r\in H}(r+A)$ と書けて,しかも右辺は非交和である.
                    $A$ が Lebesgue 可測であると仮定すると,$H$ は可算集合で Lebesgue 測度は平行移動不変なので
                    \begin{align*}
                        1
                        =\mu_\Lrm^1(G)
                        =\sum_{r\in H}\mu_\Lrm^1(r+A)
                        =\sum_{r\in H}\mu_\Lrm^1(A)
                    \end{align*}
                    となる.
                    $\mu_\Lrm^1(A)=0,>0$ のいずれであっても矛盾が起きるので,$A$ は Lebesgue 非可測である.
                    $A$ を Vitali 集合という.

                \item
                    $\Rbb^3$ における半径 $1$ の球体を有限個の集合 $A_1,\ldots,A_m$ に分割して,
                    それらを回転と平行移動によって組み替えることで,半径 $1$ の球体を二つ作ることができる.
                    これは Banach--Tarski のパラドックスと呼ばれている.
                    YouTube \cite{yts86-Z-CbaHA} に綺麗なグラフィックスを用いた解説がある (英語だが字幕もついている).

                    このとき,$A_1,\ldots,A_m$ の少なくとも一つは Lebesgue 可測ではない.
                    実際,これらがすべて Lebesgue 可測であると仮定すると,Lebesgue 測度が回転と平行移動について不変であることから
                    $4\pi^3/3=2\times(4\pi^3/3)$ がいえてしまい矛盾が起きる.
            \end{enumerate}

        \item Borel でない Lebesgue 可測集合の例\cite{ms253786} (Lusin).
            $x\in\Rbb\setminus\Qbb$ の連分数展開を
            \begin{align*}
                x=a_0+\frac{1}{a_1+\dfrac{1}{a_2+\cdots}}
            \end{align*}
            と書くと,
            \begin{align*}
                \left\{
                    x\in\Rbb\setminus\Qbb\,\middle\vert\,
                    \begin{array}{c}
                        \text{ある $0\le i_0<i_1<\cdots$ に対して}\\
                        a_{i_0}\mid a_{i_1}\mid a_{i_2}\mid\cdots
                    \end{array}
                \right\}\in\Bfrak_\Lrm^1\setminus\Bfrak^1.
            \end{align*}
    \end{enumerate}
\end{remark}

\subsection{可測関数}

位相空間論では,二つの位相空間が与えられたとき,それらの間の連続関数を考察の対象とする.
これと類似して,測度論では二つの可測空間の間の可測写像が定義される.
この節では可測写像についての基本的な事項を述べる.

\begin{definition}
    $(X,\Afrak_1),(Y,\Afrak_2)$ を可測空間とする.
    $f:X\to Y$ が可測写像 (measurable map) であるとは,任意の $B\in\Afrak_2$ に対して $f^{-1}(B)\in\Afrak_1$ となることをいう.
\end{definition}

\begin{proposition}\label{prop:Meas_prop}
    可測空間 $(X,\Afrak_1),(Y,\Afrak_2),(Z,\Afrak_3)$ に対して次が成り立つ.
    \begin{enumerate}
        \item 可測写像 $f:X\to Y,\ g:Y\to Z$ に対して,$g\circ f:X\to Z$ も可測写像.
        \item 恒等写像 $\mathrm{id}_X:X\to X$ は可測写像.
    \end{enumerate}
\end{proposition}

\begin{proof}
    定義より直ちに従う.
\end{proof}

\begin{proposition}\label{prop:Borel_functor}
    $(X,\Ocal_1),(Y,\Ocal_2)$ を位相空間,$f:X\to Y$ を連続写像とする.
    $X,Y$ を例 \ref{ex:Borel_algebra} によってそれぞれ可測空間 $(X,\Bfrak(X)),(Y,\Bfrak(Y))$ と見なすとき,
    $f$ は可測写像になる.
\end{proposition}

\begin{proof}
    \begin{align*}
        \Ocal_2
        &\subset\{B\subset Y\mid f^{-1}(B)\in\Ocal_1\}&&\because\text{$f$ は連続}\\
        &\subset\{B\subset Y\mid f^{-1}(B)\in\Bfrak(X)\}&&\because\Ocal_1\subset\Bfrak(X)
    \end{align*}
    であって,最右辺は $Y$ 上の $\sigma$-加法族であるから $\Bfrak(Y)$ を含む.
    すなわち,$B\in\Bfrak(Y)$ ならば $f^{-1}(B)\in\Bfrak(X)$.
\end{proof}

\begin{remark}
    命題 \ref{prop:Meas_prop} によって,(小さな) 可測空間を対象とし,可測写像を射とする圏が定義される.
    これを可測空間の圏といい $\mathbf{Meas}$ と書く.
    \nomenclature{$\mathbf{Meas}$}{可測空間の圏}
    $\mathbf{Top}$ を位相空間の圏とする.
    例 \ref{ex:Borel_algebra} によって位相空間から可測空間を作ることができるが,
    命題 \ref{prop:Borel_functor} により,この対応 $\mathbf{Top}\to\mathbf{Meas}$,
    \begin{itemize}
        \item $(X,\Ocal)\mapsto(X,\Bfrak(X)),$
        \item $f\mapsto f,$
    \end{itemize}
    は関手になっていることが分かる.
\end{remark}

\begin{example}
   $(X,\Afrak)$ を可測空間,$E\subset X$ とする.
   例 \ref{ex:relative_sigma_algebra} の相対 $\sigma$-加法族 $\Afrak_E$ は,
   包含写像 $\iota:E\hookrightarrow X$ が可測になるような $E$ 上の $\sigma$-加法族のうち最小のものである.
   これより特に,可測写像 $f:X\to Y$ に対して,その $E$ への制限 $f\vert_E=f\circ\iota:E\to Y$ も可測になることが分かる.
\end{example}

次に,可測写像 $f:X\to Y$ の積分を考えたい.
そのためにはもう少し空間に構造を課す必要がある.
まず,積分演算は線形性を持つべきという観点から,$Y$ (すなわち,積分の値が属する空間) はベクトル空間であってほしい.
また,積分を定義する際に極限操作が必要となるので,$Y$ は位相空間にもなっていてほしい.
従って,$Y$ が位相ベクトル空間であることを要請したい.
そのようなもののうち最も基本的な状況として,$Y$ が``数''の空間である場合を考えよう
\footnote{このように動機付けを行なったが,残念ながら
定義 \ref{def:measurable_function} に現れる $\Rbb\cup\{\pm\infty\}$ はベクトル空間ではない.
$(+\infty)+(-\infty)$ などが定義できないため.}
\footnote{他にも,$Y$ が Banach 空間の場合に適用できる積分の理論がある (Bochner 積分という).}.

\begin{definition}\label{def:measurable_function}
    $(X,\Afrak)$ を可測空間,$Y=\Rbb\cup\{\pm\infty\} $ または $\Cbb$ とする.
    例 \ref{ex:Borel_algebra}, \ref{ex:extended_real} によって $(Y,\Bfrak(Y))$ も可測空間となる.
    このときの可測写像 $f:X\to Y$ を可測関数 (measurable function) という
    \footnote{このように可測写像と可測関数を呼び分けるのがどのくらいメジャーかは分からないが,ある程度のコンセンサスは得られていると思う\cite{ms95741}.}.
    特に,$E\subset\Rbb^n$ に対して,(例 \ref{ex:relative_sigma_algebra} の記号を用いて)
    \begin{enumerate}
        \item $(X,\Afrak)=(E,(\Bfrak^n)_E)$ のとき,$f$ を Borel 可測関数という.
        \item $(X,\Afrak)=(E,(\Bfrak_\Lrm^n)_E)$ のとき,$f$ を Lebesgue 可測関数という.
    \end{enumerate}
\end{definition}

\begin{remark}
    定義 \ref{def:measurable_function} は次の意味で consistent である:
    \begin{itemize}
        \item 関数 $f:X\to\Rbb$ に対して,$f:X\to\Rbb\cup\{\pm\infty\}$ と見なしたときの可測性と
        $f:X\to\Cbb$ と見なしたときの可測性は同値になる.
    \end{itemize}
\end{remark}

\begin{remark}\label{rem:Borel_implies_Lebesgue}
    注意 \ref{rem:Borel_Lebesgue} により $\Bfrak^n\subset\Bfrak_\Lrm^n$ なので,Borel 可測関数は Lebesgue 可測である.
    特に命題 \ref{prop:Borel_functor} により,連続関数は Lebesgue 可測である.
\end{remark}

\begin{remark}[\!\!\cite{mo31603}]
    可測関数 $f:X\to Y$ に対して,$Y$ 側の $\sigma$-加法族として常に Borel 集合族 $\Bfrak(Y)$ を選んでいることに注意.
    $f$ が $\Rbb$-値 や $\Cbb\,(\simeq\Rbb^2)$-値のとき,
    $Y$ 側の $\sigma$-加法族として $\Bfrak_\Lrm^1$ や $\Bfrak_\Lrm^2$ を選ぶこともできるが,
    そのようにして定義された可測関数はあまり良い性質を持たない.
    たとえば,注意 \ref{rem:Borel_implies_Lebesgue} で見た,連続関数が可測になるという性質はもはや成立しなくなる.

    測度論で扱える関数 (または,積分できる関数) を増やしたいという文脈では,できるだけ多くの関数 $f:X\to Y$ が可測になってほしい.
    そのためには $X$ 側の $\sigma$-加法族はできるだけ大きいものを,$Y$ 側の $\sigma$-加法族はできるだけ小さいものを選択するのがよさそうである.
    この意味で,$X$ 側の $\sigma$-加法族として $(\Bfrak_\Lrm^n)_E$ を選ぶ (すなわち,Lebesgue 可測関数を考える) のは合理的であり,
    一方で $Y$ 側の $\sigma$-加法族として $\Bfrak_\Lrm^1$ などを選ぶのは指針に反する.
    また,後に \ref{sec:integration} 節で $f$ の積分を定義する際には,$Y$ の区間の $f$ による逆像が $X$ の可測集合になることを要請したい
    (命題 \ref{prop:measurable_function_characterization} と,命題 \ref{prop:approx_measurable_by_simple} の証明を参照).
    $\Bfrak(Y)$ はこの要請を満たす $\sigma$-加法族として最小のものである.
\end{remark}

可測関数の特徴づけとして,次のことが知られている (こちらを可測関数の定義とすることも多い).

\begin{proposition}\label{prop:measurable_function_characterization}
    可測空間 $(X,\Afrak)$ に対して次が成り立つ.
    \begin{enumerate}
        \item $f:X\to\Rbb\cup\{\pm\infty\}$ が可測関数であることは,次のうち任意の一つと同値.
            \begin{enumerate}
                \item 任意の $a\in\Rbb$ に対して $f^{-1}([-\infty,a))\in\Afrak$.
                \item 任意の $a\in\Rbb$ に対して $f^{-1}([-\infty,a])\in\Afrak$.
                \item 任意の $a\in\Rbb$ に対して $f^{-1}((a,+\infty])\in\Afrak$.
                \item 任意の $a\in\Rbb$ に対して $f^{-1}([a,+\infty])\in\Afrak$.
            \end{enumerate}
        \item $f:X\to\Cbb$ が可測関数であることは,$\Re f:X\to\Rbb,\ \Im f:X\to\Rbb$ が共に可測関数であることと同値.
    \end{enumerate}
\end{proposition}

\begin{proof}
    \textrm{i)} を示す.
    \begin{itemize}[align=left]
        \item[$f$ が可測 $\Rightarrow$ (a)--(d):]
            可測関数の定義から明らか.
        \item[(a) $\Rightarrow$ (b):]
            $\displaystyle f^{-1}([-\infty,a])=\bigcap_{k=1}^\infty f^{-1}\biggl(\biggl[-\infty,a+\frac{1}{k}\biggr)\biggr)\in\Afrak$ より従う.
        \item[(b) $\Rightarrow$ (c):]
            $\displaystyle f^{-1}((a,+\infty])=X\setminus f^{-1}([-\infty,a])\in\Afrak$ より従う.
        \item[(c) $\Rightarrow$ (d):]
            $\displaystyle f^{-1}([a,+\infty])=\bigcap_{k=1}^\infty f^{-1}\biggl(\biggl(a-\frac{1}{k},+\infty\biggr]\biggr)\in\Afrak$ より従う.
        \item[(d) $\Rightarrow$ (a):]
            $\displaystyle f^{-1}([-\infty,a))=X\setminus f^{-1}([a,+\infty])\in\Afrak$ より従う.
        \item[(a)--(d) $\Rightarrow$ $f$ が可測:]
            (a), (c) より $\Rbb$ の任意の開区間 $I$ に対して $f^{-1}(I)\in\Afrak$ となる.
            $\Rbb$ の開集合 $O$ は開区間の可算個の和集合として表される\footnote{
                $O$ の各点 $x$ に対して,$x$ を含み $O$ に含まれるような開区間 $I_x$ を取る.
                $\bigcup_{x\in O}I_x=O$ なので,特に $\{I_x\}_{x\in O}$ は $O$ の開被覆である.
                Lindel\"of の被覆定理 \cite[付録 \S2 定理 2]{It63} によって,これらから可算無限個を
                抜き出して,なお $O$ を被覆するようにできる.
            }ので,$f^{-1}(O)\in\Afrak$ も分かる.
            このことと \eqref{eqn:extended_real_topology} より,$\Rbb\cup\{\pm\infty\}$ の開集合系を $\widetilde\Ocal$ として
            \begin{align*}
                \widetilde\Ocal\subset\{B\subset\Rbb\cup\{\pm\infty\}\mid f^{-1}(B)\in\Afrak\}
            \end{align*}
            となる.
            この右辺は $\Rbb\cup\{\pm\infty\}$ 上の $\sigma$-加法族であるから $\Bfrak(\Rbb\cup\{\pm\infty\})$ を含む.
            ゆえに $f$ は可測関数である.
    \end{itemize}

    次に \textrm{ii)} を示す.
    TODO
\end{proof}

\begin{remark}
    Borel 可測関数や Lebesgue 可測関数の合成関数の可測性についてまとめておく.

    $E\subset\Rbb^n$ とし,$Y=\Rbb\cup\{\pm\infty\} $ または $\Cbb$ とする.
    $f:E\to\Rbb,\ g:\Rbb\to Y$ に対して,$g\circ f$ の可測性は次の表のようになる.

    \begin{table}[h]
        \centering
        \begin{tabular}{|c|c||c|}
            \hline
            $f$ & $g$ & $g\circ f$\\
            \hhline{|=|=#=|}
            Borel 可測 & Borel 可測 & Borel 可測\\
            \hline
            Lebesgue 可測 & Borel 可測 & Lebesgue 可測\\
            \hline
            Borel 可測 & Lebesgue 可測 & 何もいえない\\
            \hline
            Lebesgue 可測 & Lebesgue 可測 & 何もいえない\\
            \hline
        \end{tabular}
    \end{table}

    特に,$g$ が Lebesgue 可測であることしか分かっていないときには,合成関数の可測性については何も保証されないことに注意.
    $1,2$ 行目は可測関数の定義と $\Bfrak^n\subset\Bfrak_\Lrm^n$ から直ちに従う.
    $3$ 行目は反例が \cite[pp.72--73]{It63} にある.
    $4$ 行目は $3$ 行目から直ちに従う.
\end{remark}

\begin{theorem}\label{thm:elementary_measuable_functions}
    $(X,\Afrak)$ を可測空間とする.
    \begin{enumerate}
        \item 可測関数 $f:X\to\Cbb,\ g:X\to\Cbb$ と $\alpha\in\Cbb,\ p>0$ に対して,$f+g,\ \alpha f,\ fg,\ |f|^p$ は可測である.
        \item 可測関数列 $(f_i:X\to\Rbb\cup\{\pm\infty\})_{i=1}^\infty$ に対して,$\displaystyle\sup_{i\ge1}f_i,\ \inf_{i\ge1}f_i$ は可測である.
    \end{enumerate}
\end{theorem}

\begin{theorem}\label{thm:measurable_function_pointwise_convergence}
    $(X,\Afrak)$ を可測空間,$Y=\Rbb\cup\{\pm\infty\}$ または $\Cbb$ とする.
    可測関数列 $(f_i:X\to Y)_{i=1}^\infty$ が関数 $f:X\to Y$ に $X$ 上で各点収束するなら,$f$ も可測である.
\end{theorem}

\begin{remark}
    定理 \ref{thm:measurable_function_pointwise_convergence} は
    可測関数の空間が各点収束について閉じていることを主張しており,
    これは非常に多くの可測関数が存在することを示唆している (と思う).
    このことは連続関数の空間との大きな違いである.
    連続関数列の極限関数が連続になることを保証するためには,各点収束では不十分で,一様収束の概念が必要だった.
\end{remark}

\subsection{Lebesgue 積分}\label{sec:integration}

この節では,一般の測度空間上の可測関数に対して積分を定義し,その基本的な性質を調べる.

\begin{definition}
    $X$ を集合とする.
    $E\subset X$ に対して,
    \begin{align*}
        \chi_E(x)\defeq\begin{cases}
            1&\text{if $x\in E$}\\
            0&\text{otherwise}
        \end{cases}
    \end{align*}
    で定義される関数 $\chi_E:X\to\{0,1\}$ を $E$ の定義関数
    (または指示関数 (indicator function),特性関数 (characteric function)) という.
    \nomenclature{$\chi_E$}{集合 $E$ の定義関数}
\end{definition}

\begin{definition}
    $X$ を集合とする.
    $X$ 上の実数値関数であって有限種類の値しか取らないもの,すなわち,
    $X$ の互いに素な部分集合 $E_1,\ldots,E_k$ と相異なる $\alpha_1,\ldots,\alpha_k\in\Rbb\setminus\{0\}$ を用いて
    \begin{align}
        f(x)=\sum_{i=1}^k\alpha_i\chi_{E_i}(x)
        \label{eqn:simple_function}
    \end{align}
    と表される関数 $f:X\to\Rbb$ を $X$ 上の単関数 (simple function) という.
\end{definition}

\begin{proposition}\label{prop:measurability_of_simple_function}
    $(X,\Afrak)$ を可測空間とする.
    単関数 \eqref{eqn:simple_function} が可測関数であることは $E_1,\ldots,E_k\in\Afrak$ と同値である.
\end{proposition}

\begin{proof}
    明らか.
\end{proof}

次の命題は,非負値可測関数が可測な非負値単関数で近似できることを主張する.

\begin{proposition}\label{prop:approx_measurable_by_simple}
    $(X,\Afrak)$ を可測空間,$f:X\to\Rbb\cup\{\pm\infty\}$ を非負値可測関数とする.
    このとき,非負値可測単関数の列 $(f_i:X\to\Rbb)_{i=1}^\infty$ であって $f$ に $X$ 上で各点収束するものが存在する.
\end{proposition}

\begin{proof}
    \begin{align*}
        f_i(x)=\begin{cases}
            \dfrac{j-1}{2^i}&\text{ある $j\in\{1,2,\ldots,i2^i\}$ に対して $\dfrac{j-1}{2^i}\le f(x)<\dfrac{j}{2^i}$ のとき}\\
            \quad i&\text{$f(x)\ge i$ のとき}
        \end{cases}
    \end{align*}
    と定める.
    命題 \ref{prop:measurable_function_characterization} より
    \begin{align*}
        &\left\{x\in X\,\middle\vert\,f_i(x)=\dfrac{j-1}{2^i}\right\}
        =f^{-1}\biggl(\biggl[\dfrac{j-1}{2^i},\dfrac{j}{2^i}\biggr)\biggr)\in\Afrak\\
        &\{x\in X\mid f_i(x)=i\}=f^{-1}([i,+\infty])\in\Afrak
    \end{align*}
    となるので,命題 \ref{prop:measurability_of_simple_function} より各 $f_i$ は可測である.
    また,任意の $x\in X$ に対して $(f_i(x))_{i=1}^\infty$ は単調増加するので,
    ($+\infty$ も許せば) $\displaystyle\lim_{i\to\infty}f_i(x)$ が存在.

    $\displaystyle\lim_{i\to\infty}f_i(x)=f(x)$ を示す.
    $f(x)<+\infty$ のとき,$i>f(x)$ なる任意の $i$ に対して $|f_i(x)-f(x)|\le1/2^i$ となるのでよい.
    一方,$f(x)=+\infty$ のときは $f_i(x)=i$ なのでよい.
\end{proof}

いよいよ可測関数の積分を定義する.
これは三つの段階からなる.

\begin{definition}\label{def:Lebesgue_integral}
    $(X,\Afrak,\mu)$ を測度空間とする.
    \begin{enumerate}
        \item $f:X\to\Rbb\cup\{\pm\infty\}$ を非負値可測単関数とする.
            $f$ は $E_1,\ldots,E_k\in\Afrak$ と相異なる正の実数 $\alpha_1,\ldots,\alpha_k$ を用いて
            \eqref{eqn:simple_function} と一意に表される.
            このとき,$f$ の積分を
            \begin{align*}
                \int_Xf(x)\,\drm\mu(x)\defeq\sum_{i=1}^k\alpha_i\mu(E_i)
            \end{align*}
            と定める.
        \item $f:X\to\Rbb\cup\{\pm\infty\}$ を非負値可測関数とする.
            命題 \ref{prop:approx_measurable_by_simple} より,
            $f$ に各点収束する非負値単関数列 $(f_i:X\to\Rbb)_{i=1}^\infty$ が存在する.
            このとき,$f$ の積分を
            \begin{align*}
                \int_Xf(x)\,\drm\mu(x)\defeq\lim_{i\to\infty}\int_Xf_i(x)\,\drm\mu(x)
            \end{align*}
            と定める (値が $+\infty$ でもよい).
            これは well-defined である \cite[pp.74--77]{It63}.
        \item
            \begin{enumerate}
                \item
                    $f:X\to\Rbb\cup\{\pm\infty\}$ を可測関数とする.
                    $f^+(x)\defeq\max\{f(x),0\},\ f^-(x)\defeq\max\{-f(x),0\}$ とおくと,
                    定理 \ref{thm:elementary_measuable_functions} より $f^+,f^-$ は非負値可測関数であって,
                    $f=f^+-f^-$ と書ける.
                    \begin{align}
                        \int_Xf^+(x)\,\drm\mu(x),\quad
                        \int_Xf^-(x)\,\drm\mu(x)
                        \label{eqn:f_pm_integral}
                    \end{align}
                    の少なくとも一方が有限であることを仮定する.
                    このとき,$f$ の積分を
                    \begin{align*}
                        \int_Xf(x)\,\drm\mu(x)\defeq\int_Xf^+(x)\,\drm\mu(x)-\int_Xf^-(x)\,\drm\mu(x)
                    \end{align*}
                    と定める (値が $\pm\infty$ でもよい).
                    \eqref{eqn:f_pm_integral} の両方が有限であるとき,$f$ は可積分 (integrable) であるという.
                \item
                    $f:X\to\Cbb$ を可測関数とする.
                    命題 \ref{prop:measurable_function_characterization} より $\Re f,\Im f$ は可測関数である.
                    $\Re f,\Im f$ が \textrm{iii)} (a) の意味で可積分であると仮定する.
                    このとき,$f$ は可積分であるといい,$f$ の積分を
                    \begin{align*}
                        \int_Xf(x)\,\drm\mu(x)\defeq\int_X\Re f(x)\,\drm\mu(x)+\im\int_X\Im f(x)\,\drm\mu(x)
                    \end{align*}
                    と定める.
            \end{enumerate}
    \end{enumerate}
\end{definition}

このように定義される,測度空間上の可測関数の積分を Lebesgue 積分という.
ただし,特に Lebesgue 測度に関する積分
(すなわち,測度空間として $(\Rbb^n,\Bfrak_\Lrm^n,\mu_\Lrm^n)$ (またはその可測部分集合への制限) を選んだ場合)
のことを指して Lebesgue 積分という語を使うこともある.
このときの積分を
\begin{align*}
    \int_{\Rbb^n}f(x)\,\drm x\defeq\int_{\Rbb^n}f(x)\,\drm\mu_\Lrm^n(x)
\end{align*}
と表記する.

Lebesgue 積分の基本的な性質をまとめておく.

\begin{proposition}
    $(X,\Afrak,\mu)$ を測度空間,$Y=\Rbb\cup\{\pm\infty\} $ または $\Cbb$ とする.
    可測関数 $f:X\to Y$ が可積分であることは,定義 \ref{def:Lebesgue_integral} \textrm{ii)} の意味で
    \begin{align*}
        \int_X|f(x)|\,\drm\mu(x)<+\infty
    \end{align*}
    であることと同値である.
\end{proposition}

\begin{theorem}
    $(X,\Afrak,\mu)$ を測度空間,$Y=\Rbb\cup\{\pm\infty\} $ または $\Cbb$ とする.
    可積分関数 $f:X\to Y$ に対して
    \begin{align*}
        \left\lvert\int_Xf(x)\,\drm\mu(x)\right\rvert
        \le\int_X|f(x)|\,\drm\mu(x)
    \end{align*}
    が成り立つ.
\end{theorem}

\begin{proof}
    \cite[定理 12.1]{It63} を参照.
\end{proof}

\begin{theorem}
    $(X,\Afrak,\mu)$ を測度空間とする.
    可積分関数 $f:X\to\Cbb,\ g:X\to\Cbb$ と $\alpha,\beta\in\Cbb$ に対して,
    $\alpha f+\beta g$ も可積分であって
    \begin{align*}
        \int_X(\alpha f(x)+\beta g(x))\,\drm\mu(x)
        =\alpha\int_Xf(x)\,\drm\mu(x)+\beta\int_Xg(x)\,\drm\mu(x)
    \end{align*}
    が成り立つ.
\end{theorem}

\begin{proof}
    \cite[定理 12.3, 系 1]{It63} を参照.
\end{proof}

\begin{definition}
    $(X,\Afrak,\mu)$ を測度空間とする.
    $X$ の点 $x$ に関する命題 $P(x)$ に対して,
    ある零集合 $N\in\Afrak$ (すなわち,$\mu(N)=0$) が存在して,任意の $x\in X\setminus N$ で $P(x)$ が成立するとき,
    $P$ は $\mu$ について $X$ 上ほとんどいたる所 (almost everywhere) 成立するといい,
    $P(x)\ \textrm{$\mu$-a.e.\ $x\in X$}$ と書く.
    文脈から明らかなときは $P\ \textrm{a.e.\ $X$}$ などと略記する.
    \nomenclature{$\textrm{a.e.}$}{ほとんどいたる所}
\end{definition}

\begin{theorem}
    $(X,\Afrak,\mu)$ を測度空間とする.
    \begin{enumerate}
        \item 可積分関数 $f:X\to\Cbb,\ g:X\to\Cbb$ に対して,
            $f(x)=g(x)\ \textrm{$\mu$-a.e.\ $x\in X$}$ ならば
            \begin{align*}
                \int_Xf(x)\drm\mu(x)=\int_Xg(x)\,\drm\mu(x)
            \end{align*}
            が成り立つ.
        \item 可積分関数 $f:X\to\Cbb$ に対して,$f(x)=0\ \textrm{$\mu$-a.e.\ $x\in X$}$ であることは
            \begin{align*}
                \int_X|f(x)|\,\drm\mu(x)=0
            \end{align*}
            と同値である.
    \end{enumerate}
\end{theorem}

\begin{proof}
    \cite[定理 12.3, 系 2]{It63} を参照.
\end{proof}

\begin{example}
    Riemann 可積分な関数は Lebesgue 積分の意味でも可積分であって積分値は等しい.
    一方で,広義 Riemann 積分まで含めるとこれは必ずしも成り立たない.
    $\Rbb$ 上の関数 $\sin x/x$ (ただし,$x=0$ での値は $1$ と定める) は
    \begin{align*}
        \int_{\Rbb}\left\lvert\frac{\sin x}{x}\right\rvert\drm x=+\infty
    \end{align*}
    となるので Lebesgue 積分の意味で可積分ではないが,
    \begin{align*}
        \lim_{\substack{L\to-\infty\\R\to+\infty}}\int_{L}^R\frac{\sin x}{x}\,\drm x=\pi
    \end{align*}
    となるので広義 Riemann 可積分である.
\end{example}

\subsection{収束定理と積分の順序交換に関する定理}



\begin{thebibliography}{99}
	\bibitem{Fo99} Folland, G.B., \textit{Real Analysis: Modern Techniques and thier Applications} (2nd ed.), John Wiley, 1999.
	\bibitem{It63} 伊藤清三,ルベーグ積分入門,裳華房, 1963.
	\bibitem{mo31603} Why do probabilists take random variables to be Borel (and not Lebesgue) measurable? - mathoverflow StackExchange,
		\url{https://mathoverflow.net/questions/31603/}
	\bibitem{ms70880} Cardinality of Borel sigma algebra - MATHEMATICS StackExchange,\\
		\url{https://math.stackexchange.com/questions/70880/}
	\bibitem{ms95741} Is there any difference between mapping and function? - MATHEMATICS StackExchange,\\
		\url{https://math.stackexchange.com/questions/95741/}
	\bibitem{ms253786} Intuition behind the non-Borel Lusin example - MATHEMATICS StackExchange,\\
		\url{https://mathoverflow.net/questions/253786/}
	\bibitem{ms308856} Why is Lebesgue measure theory asymmetric? - MATHEMATICS StackExchange,\\
		\url{https://mathoverflow.net/questions/308856/}
	\bibitem{yts86-Z-CbaHA} Vsause, The Banach–Tarski Paradox - YouTube,
		\url{https://www.youtube.com/watch?v=s86-Z-CbaHA}
	\bibitem{mpmi} 入門テキスト「測度と積分」- MATHPEDIA,
		\href{https://math.jp/wiki/%E5%85%A5%E9%96%80%E3%83%86%E3%82%AD%E3%82%B9%E3%83%88%E3%80%8C%E6%B8%AC%E5%BA%A6%E3%81%A8%E7%A9%8D%E5%88%86%E3%80%8D}{https://math.jp/wiki/入門テキスト「測度と積分」}
\end{thebibliography}


\printnomenclature

\end{document}
