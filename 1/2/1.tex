\subsubsection{可測写像}

\begin{definition}
    $(X_1,\Afrak_1),(X_2,\Afrak_2)$ を可測空間とする.
    $f:X_1\to X_2$ が可測写像 (measurable map) であるとは,任意の $A\in\Afrak_2$ に対して $f^{-1}(A)\in\Afrak_1$ となることをいう.
\end{definition}

\begin{proposition}\label{prop:Meas_prop}
    $(X_1,\Afrak_1),(X_2,\Afrak_2),(X_3,\Afrak_3)$ を可測空間とするとき,次が成り立つ.
    \begin{enumerate}
        \item 可測写像 $f:X_1\to X_2,\ g:X_2\to X_3$ に対して,$g\circ f:X_1\to X_3$ も可測写像.
        \item 恒等写像 $\mathrm{id}_{X_1}:{X_1}\to{X_1}$ は可測写像.
    \end{enumerate}
\end{proposition}

\begin{proof}
    明らか.
\end{proof}

\begin{example}\label{ex:initial_sigma_algebra}
    \ref{sec:measurable_space} 節で見た $\sigma$-加法族の構成のうちいくつかは,ある写像の族を可測にするような最小の $\sigma$-加法族として統一的に扱うことができる
    \footnote{位相空間論における initial topology の概念によく似ている.}.
    \begin{enumerate}
        \item
            $(X,\Afrak)$ を可測空間,$E\subset X$ とする.
            例 \ref{ex:relative_sigma_algebra} の相対 $\sigma$-加法族 $\Afrak_E$ は,
            包含写像 $\iota:E\to X$ が可測になるような $E$ 上の $\sigma$-加法族のうち最小のものである.
            すなわち,
            \begin{align*}
                \Afrak_E
                =\{\iota^{-1}(A)\mid A\in\Afrak\}
                =\sigma(\{\iota^{-1}(A)\mid A\in\Afrak\}).
            \end{align*}
        \item
            $((X_\lambda,\Afrak_\lambda))_{\lambda\in\Lambda}$ を可測空間の族とする.
            例 \ref{ex:product_sigma_algebra} の直積 $\sigma$-加法族 $\bigotimes_{\lambda\in\Lambda}\Afrak_\lambda$ は,
            自然な射影 $\pi_\lambda:\prod_{\lambda\in\Lambda}X_\lambda\to X_\lambda$ が可測になるような
            $\prod_{\lambda\in\Lambda}X_\lambda$ 上の $\sigma$-加法族のうち最小のものである.
        \end{enumerate}
\end{example}

\begin{proposition}\label{prop:Meas_product}
    $(X,\Afrak)$ を可測空間,$(Y,\Cfrak)$ を可測空間の族 $((Y,\Cfrak_\lambda))_{\lambda\in\Lambda}$ の直積可測空間とする.
    自然な射影を $\pi_\lambda:Y\to Y_\lambda$ と置く.
    このとき,写像 $f=(f_\lambda)_{\lambda\in\Lambda}:X\to Y$ に対して次は同値である.
    \begin{enumerate}
        \item $f:X\to Y$ は可測写像である.
        \item 任意の $\lambda\in\Lambda$ に対して $f_\lambda=\pi_\lambda\circ f:X\to Y_\lambda$ は可測写像である.
    \end{enumerate}
    \begin{equation*}
        \begin{tikzcd}
            X\arrow[rd,"f_\lambda"]\arrow[d,"f"']\\
            Y\arrow[r,"\pi_\lambda"]&Y_\lambda
        \end{tikzcd}
    \end{equation*}
\end{proposition}

\begin{proof}
    \textrm{i)} $\Rightarrow$ \textrm{ii)} は,
    例 \ref{ex:initial_sigma_algebra} \textrm{ii)} より $\pi_\lambda$ が可測になることから直ちに従う.
    \textrm{ii)} $\Rightarrow$ \textrm{i)} を示す.
    任意の $C\in\Cfrak_\lambda$ に対して
    \begin{align*}
        f^{-1}(\pi_\lambda^{-1}(C))
        =(\pi_\lambda\circ f)^{-1}(C)
        =f_\lambda^{-1}(C)\in\Afrak
    \end{align*}
    となるので,
    \begin{align*}
        \{\pi_\lambda^{-1}(C)\mid C\in\Cfrak_\lambda,\ \lambda\in\Lambda\}
        \subset\{C\subset Y\mid f^{-1}(C)\in\Afrak\}.
    \end{align*}
    右辺は $Y$ 上の $\sigma$-加法族であるから $\Cfrak$ を含む.
    ゆえに $f$ は可測.
\end{proof}

\begin{remark}
    命題 \ref{prop:Meas_prop} によって,(小さな) 可測空間を対象とし,可測写像を射とする圏が定義される.
    これを可測空間の圏といい $\mathbf{Meas}$ と書く.
    \nomenclature{$\mathbf{Meas}$}{可測空間の圏}
    また,命題 \ref{prop:Meas_product} によって直積可測空間は $\mathbf{Meas}$ における積になることが分かる.
\end{remark}

\begin{proposition}\label{prop:conti_implies_measurable}
    $(X_1,\Ocal_1),(X_2,\Ocal_2)$ を位相空間,$f:X_1\to X_2$ を連続写像とする.
    $X_1,X_2$ を例 \ref{ex:Borel_algebra} によって可測空間 $(X_1,\Bfrak(X_1)),(X_2,\Bfrak(X_2))$ と見なすとき,
    $f$ は可測写像になる.
\end{proposition}

\begin{proof}
    \begin{align*}
        \Ocal_2
        &\subset\{A\subset X_2\mid f^{-1}(A)\in\Ocal_1\}&&\because\text{$f$ は連続}\\
        &\subset\{A\subset X_2\mid f^{-1}(A)\in\Bfrak(X_1)\}&&\because\Ocal_1\subset\Bfrak(X_1)
    \end{align*}
    であって,最右辺は $X_2$ 上の $\sigma$-加法族であるから $\Bfrak(X_2)$ を含む.
    すなわち,$A\in\Bfrak(X_2)$ ならば $f^{-1}(A)\in\Bfrak(X_1)$.
\end{proof}

\begin{remark}\label{rem:Borel_functor}
    $\mathbf{Top}$ を位相空間の圏とする.
    例 \ref{ex:Borel_algebra} によって位相空間から可測空間を作ることができるが,
    命題 \ref{prop:conti_implies_measurable} により,この対応 $\mathbf{Top}\to\mathbf{Meas}$,
    \begin{itemize}
        \item $(X,\Ocal)\mapsto(X,\Bfrak(X)),$
        \item $f\mapsto f,$
    \end{itemize}
    は関手であることが分かる.
\end{remark}

次の二つの命題は,相対 $\sigma$-加法族や直積 $\sigma$-加法族を取る構成と注意 \ref{rem:Borel_functor} の関手 $\mathbf{Top}\to\mathbf{Meas}$
との関係を述べている.

\begin{proposition}
    $(X,\Ocal)$ を位相空間,$E\subset X$ とする.
    包含写像を $\iota:E\to X$ と置く.
    $E$ は相対位相 $\Ocal_E\defeq\{O\cap E\mid O\in\Ocal\}$ について位相空間となる.
    \nomenclature{$\Ocal_E$}{$\Ocal$ から誘導される $E$ 上の相対位相}
    このとき,$E$ を可測空間と見なす方法が $2$ 通りありうる:
    \begin{itemize}
        \item $(E,\Ocal_E)$ に対して,例 \ref{ex:Borel_algebra} のやり方で $E$ 上の Borel 集合族をとる.
        \item $(X,\Bfrak(X))$ に対して,例 \ref{ex:relative_sigma_algebra} のやり方で $E$ 上の相対 $\sigma$-加法族をとる.
    \end{itemize}
    これら二つの構成は一致する.
    すなわち,$\Bfrak(E)=\Bfrak(X)_E$ が成り立つ.
    \begin{equation*}
        \begin{tikzcd}
            (X,\Ocal)\arrow[r,mapsto,"\text{Borel}"]\arrow[d,mapsto,"\text{relative topology}"']&
            (X,\Bfrak(X))\arrow[d,mapsto,"\text{relative $\sigma$-algebra}"]\\
            (E,\Ocal_E)\arrow[r,mapsto,"\text{Borel}"]&(E,\Bfrak(X)_E)
        \end{tikzcd}
    \end{equation*}
\end{proposition}

\begin{proof}
    相対位相の定義によって $\iota:E\to X$ は連続になるので,
    命題 \ref{prop:conti_implies_measurable} より $\iota:(E,\Bfrak(E))\to(X,\Bfrak(X))$ は可測.
    従って,例 \ref{ex:initial_sigma_algebra} \textrm{i)} より $\Bfrak(X)_E\subset\Bfrak(E)$ が分かる.
    一方,任意の $O\in\Ocal$ に対して
    \begin{align*}
        O\cap E=\iota^{-1}(O)\in\{\iota^{-1}(A)\mid A\in\Bfrak(X)\}=\Bfrak(X)_E
    \end{align*}
    となるので,$\Ocal_E\subset\Bfrak(X)_E$.
    従って $\Bfrak(E)\subset\Bfrak(X)_E$ を得る.
\end{proof}

\begin{proposition}\label{prop:product_compatibility}
    $((X_\lambda,\Ocal_\lambda))_{\lambda\in\Lambda}$ を位相空間の族とする.
    直積集合を $X=\prod_{\lambda\in\Lambda}X_\lambda$,
    自然な射影を $\pi_\lambda:X\to X_\lambda$ と置く.
    $X$ は直積位相 $\Ocal$ について位相空間となる.
    このとき,$X$ を可測空間と見なす方法が $2$ 通りありうる:
    \begin{itemize}
        \item $(X,\Ocal)$ に対して,例 \ref{ex:Borel_algebra} のやり方で $X$ 上の Borel 集合族をとる.
        \item $((X_\lambda,\Bfrak(X_\lambda)))_{\lambda\in\Lambda}$ に対して,
            例 \ref{ex:product_sigma_algebra} のやり方で $X$ 上の直積 $\sigma$-加法族をとる.
    \end{itemize}
    これら二つの構成の関係について,次のことが成り立つ.
    \begin{enumerate}
        \item $\bigotimes_{\lambda\in\Lambda}\Bfrak(X_\lambda)\subset\Bfrak(X)$ が成り立つ.
        \item $\Lambda$ は高々可算な集合であって,各 $\lambda\in\Lambda$ に対して $X_\lambda$ は第二可算公理を満たすとする.
            このとき,$X$ も第二可算公理を満たし,$\Bfrak(X)\subset\bigotimes_{\lambda\in\Lambda}\Bfrak(X_\lambda)$ が成り立つ
            \footnote{
                位相が任意個の合併について閉じているのに対して,$\sigma$-加法族は可算個の合併についてしか閉じていない.
                証明において,第二可算公理はこのギャップを埋める (すなわち,任意個の合併が実は可算個の合併として書けることを保証する) ために用いられる.
            }.
    \end{enumerate}
    \begin{equation*}
        \begin{tikzcd}
            ((X_i,\Ocal_i))_{i=1}^\infty\arrow[r,mapsto,"\text{Borel}"]\arrow[d,mapsto,"\text{product topology}"']&
            ((X_i,\Bfrak(X_i)))_{i=1}^\infty\arrow[d,mapsto,"\text{product $\sigma$-algebra}"]\\
            \displaystyle\biggl(\prod_{i=1}^\infty X_i,\Ocal\biggr)\arrow[r,mapsto,"\text{Borel}"]&
            \displaystyle\biggl(\prod_{i=1}^\infty X_i,\bigotimes_{i=1}^\infty\Bfrak(X_i)\biggr)
        \end{tikzcd}
    \end{equation*}
\end{proposition}

\begin{proof}
    各 $\lambda\in\Lambda$ に対して,直積位相の定義によって $\pi_\lambda:X\to X_\lambda$ は連続になるので,
    命題 \ref{prop:conti_implies_measurable} より $\pi_\lambda:(X,\Bfrak(X))\to(X_\lambda,\Bfrak(X_\lambda))$ は可測.
    従って,例 \ref{ex:initial_sigma_algebra} \textrm{ii)} より \textrm{i)} が分かる.

    \textrm{ii)} を示す.
    $\Ucal_\lambda$ を $X_\lambda$ の高々可算な開基として
    \begin{align*}
        \Ucal
        =\{\pi_{\lambda_1}^{-1}(U_1)\cap\cdots\cap\pi_{\lambda_k}^{-1}(U_k)
        \mid U_i\in\Ucal_{\lambda_i},\lambda_i\in\Lambda\ \text{for}\ i=1,2,\ldots,k;\ k<+\infty\}
    \end{align*}
    と置く.
    $\Ucal$ は $X$ の開基である.
    実際,任意の $O_i\in\Ocal_{\lambda_i},\lambda_i\in\Lambda,\ i=1,2,\ldots,k,$ に対して,
    各 $O_i$ は $\Ucal_{\lambda_i}$ の元の和集合で表されるので,
    \begin{align}
        \pi_{\lambda_1}^{-1}(O_1)\cap\cdots\cap\pi_{\lambda_k}^{-1}(O_k)
        \label{eqn:product_open_basis}
    \end{align}
    は $\Ucal$ の元の和集合で表される.
    また,$\Lambda,\,\Ucal_{\lambda_i}$ は高々可算であって
    \begin{align*}
        \Ucal
        =\bigcup_{k=1}^\infty\{\pi_{\lambda_1}^{-1}(U_1)\cap\cdots\cap\pi_{\lambda_k}^{-1}(U_k)
        \mid U_i\in\Ucal_{\lambda_i},\lambda_i\in\Lambda\ \text{for}\ i=1,2,\ldots,k\}
    \end{align*}
    と書けるので,$\Ucal$ も高々可算である.
    ゆえに,$X$ は第二可算である.

    任意の $O\in\Ocal_\lambda$ に対して
    \begin{align*}
        \pi_\lambda^{-1}(O)\in\{\pi_\lambda^{-1}(A)\mid A\in\Bfrak(X_\lambda)\}\subset\bigotimes_{\lambda\in\Lambda}\Bfrak(X_\lambda)
    \end{align*}
    となる.
    注意 \ref{rem:measurable_space_other_axioms} \textrm{ii)} より,
    \eqref{eqn:product_open_basis} の形の元も $\bigotimes_{\lambda\in\Lambda}\Bfrak(X_\lambda)$ に属する.
    $\Ucal$ は高々可算な $X$ の開基だったので,\ref{item:sa3} より $X$ の任意の開集合も $\bigotimes_{\lambda\in\Lambda}\Bfrak(X_\lambda)$ に属する.
    ゆえに,$\Bfrak(X)\subset\bigotimes_{\lambda\in\Lambda}\Bfrak(X_\lambda)$ を得る.
\end{proof}

\begin{corollary}
    $\Bfrak^n=\underbrace{\Bfrak^1\otimes\cdots\otimes\Bfrak^1}_{\text{$n$ 個}}$.
\end{corollary}

\begin{proof}
    命題 \ref{prop:product_compatibility} を $\Lambda=\{1,2,\ldots,n\},X_\lambda=\Rbb$ として適用すればいい.
\end{proof}
