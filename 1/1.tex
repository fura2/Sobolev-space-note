\subsection{測度}

集合に対して,``その部分集合の大きさを測る''機構を付与することで測度空間の概念が得られる.
また,その前段階として現れる可測空間もそれ自身が重要な対象である.
この節では,これらの定義から始めて,外測度を用いた測度の構成法について述べる.
また,具体例として Lebesgue 測度を構成する.

\begin{definition}
    $X$ を集合とする.
    次の $3$ つの条件を満たす $\Afrak\subset2^X$ を $X$ 上の $\sigma$-加法族 ($\sigma$-algebra) という.
    \begin{enumerate}[label=\textsf{(SA\arabic*)},align=left]
        \item\label{item:sa1} $\emptyset\in\Afrak$.
        \item\label{item:sa2} 任意の $A\in\Afrak$ に対して $A^\crm\in\Afrak$.\quad (ここで,$A^\crm\defeq X\setminus A$)
        \item\label{item:sa3} 任意の $(A_i)_{i=1}^\infty\subset\Afrak$ に対して $\displaystyle\bigcup_{i=1}^\infty A_i\in\Afrak$.
    \end{enumerate}
    \nomenclature{$A^\crm$}{集合 $A$ の補集合}
    組 $(X,\Afrak)$ を可測空間 (measurable space),$\Afrak$ の元を可測集合 (measurable set) という.
\end{definition}

\begin{remark}
    可測空間 $(X,\Afrak)$ に対して次が成り立つ.
    \begin{enumerate}
        \item $X\in\Afrak$.
        \item 任意の $(A_i)_{i=1}^\infty\subset\Afrak$ に対して $\displaystyle\bigcap_{i=1}^\infty A_i\in\Afrak$.
    \end{enumerate}
    前者は \ref{item:sa1} と \ref{item:sa2} から,後者は \ref{item:sa2} と \ref{item:sa3} から従う.
\end{remark}

\begin{example}
    $\Afrak=\{\emptyset,X\},2^X$ はいずれも $X$ 上の $\sigma$-加法族になる.
\end{example}

\begin{example}\label{ex:relative_sigma_algebra}
    可測空間 $(X,\Afrak)$ と $E\subset X$ に対して,
    \begin{align*}
        \Afrak_E\defeq\{A\cap E\mid A\in\Afrak\}
    \end{align*}
    は $E$ 上の $\sigma$-加法族であって,$E$ は自然に可測空間 $(E,\Afrak_E)$ となる.
    $\Afrak_E$ を $\Afrak$ から誘導される $E$ 上の相対 $\sigma$-加法族という.
    \nomenclature{$\Afrak_E$}{$\Afrak$ から誘導される $E$ 上の相対 $\sigma$-加法族}
    ここでは $E\in\Afrak$ を仮定していないことに注意 (例 \ref{ex:restriction_of_measure} も参照).
\end{example}

\begin{example}
    集合 $X$ 上の $\sigma$-加法族の族 $\{\Afrak_\lambda\}_{\lambda\in\Lambda}$ に対して,
        $\bigcap_{\lambda\in\Lambda}\Afrak_\lambda$
    も $X$ 上の $\sigma$-加法族になる.

    $S\subset 2^X$に対して,$S$ を含む (包含関係について) 最小の $\sigma$-加法族が
    \begin{align*}
        \sigma[S]\defeq\bigcap_{\substack{\text{$\Afrak$: $\sigma$-algebra on $X$}\\S\subset\Afrak}}\Afrak
    \end{align*}
    として一意に定まる.
    これを $S$ によって生成される $X$ 上の $\sigma$-加法族という.
    \nomenclature{$\sigma[S]$}{$S\subset2^X$ によって生成される $X$ 上の $\sigma$-加法族}
\end{example}

\begin{example}\label{ex:Borel_algebra}
    位相空間 $(X,\Ocal)$ に対して,$X$ の開集合系によって生成される $\sigma$-加法族
    $\Bfrak(X)\defeq\sigma[\Ocal]$ を Borel 集合族 (Borel algebra),
    $\Bfrak(X)$ の元を Borel 集合 (Borel set) という.
    また,$\Bfrak^n\defeq\Bfrak(\Rbb^n)$ と置く.
    \nomenclature{$\Bfrak(X)$}{位相空間 $X$ 上の Borel 集合族}
    \nomenclature{$\Bfrak^n$}{$\Rbb^n$ 上の Borel 集合族}
\end{example}

\begin{example}
    $(X,\Ocal)$ を位相空間,$E\subset X$ とする.
    $E$ は相対位相 $\Ocal_E\defeq\{O\cap E\mid O\in\Ocal\}$ について位相空間となる.
    \nomenclature{$\Ocal_E$}{$\Ocal$ から誘導される $E$ 上の相対位相}
    このとき,$E$ を可測空間と見なす方法が $2$ 通りありうる:
    \begin{enumerate}
        \item $(E,\Ocal_E)$ に対して,例 \ref{ex:Borel_algebra} のやり方で $E$ 上の Borel 集合族をとる.
        \item $(X,\Bfrak(X))$ に対して,例 \ref{ex:relative_sigma_algebra} のやり方で $E$ 上の相対 $\sigma$-加法族をとる.
    \end{enumerate}
    実はこれら二つの構成は一致する.
    すなわち,$\sigma[\Ocal_E]=\sigma[\Ocal]_E$ が成り立つ.
    実際,
    \begin{align*}
        \Ocal_E
        =\{O\cap E\mid O\in\Ocal\}
        \subset\{A\cap E\mid A\in\sigma[\Ocal]\}
        =\sigma[\Ocal]_E
    \end{align*}
    で,最右辺は $E$ 上の $\sigma$-加法族なので $\sigma[\Ocal_E]\subset\sigma[\Ocal]_E$ となる.
    一方,
    \begin{align*}
        \Ocal
        \subset\{A\subset X\mid A\cap E\in\sigma[\Ocal_E]\}
    \end{align*}
    の右辺は $\sigma$-加法族となるので $\sigma[\Ocal]$ を含む.
    従って,任意の $A\in\sigma[\Ocal]$ に対して $A\cap E\in\sigma[\Ocal_E]$,
    すなわち $\sigma[\Ocal]_E\subset\sigma[\Ocal_E]$ となる.
    \begin{align*}
        \begin{CD}
            (X,\Ocal) @>{\text{Borel}}>> (X,\Bfrak(X))\\
            @V{\text{relative topology}}VV @VV{\text{relative $\sigma$-algebra}}V\\
            (E,\Ocal_E) @>{\text{Borel}}>> (E,\Bfrak(E))
        \end{CD}
    \end{align*}
\end{example}

\begin{example}\label{ex:extended_real}
    実数体 $\Rbb$ に形式的な元 $+\infty,-\infty$ を加えた集合 $\Rbb\cup\{\pm\infty\}$ を考える.
    $a\in\Rbb$ に対して,順序関係を $-\infty<a<+\infty$,演算を
    \begin{align*}
        &(\pm\infty)+a=a+(\pm\infty)\defeq\pm\infty,\quad
        (\pm\infty)+(\pm\infty)\defeq\pm\infty,\\
        &(\pm\infty)-a=a-(\pm\infty)\defeq\mp\infty,\quad
        (\pm\infty)-(\mp\infty)\defeq\pm\infty,\\
        &-(+\infty)\defeq-\infty,\quad
        -(-\infty)\defeq+\infty\\
        &a\cdot(\pm\infty)=(\pm\infty)\cdot a\defeq\begin{cases}
            \pm\infty&\text{($0<a<+\infty$ のとき)}\\
            \mp\infty&\text{($-\infty<a<0$ のとき)},
        \end{cases}\\
        &(\pm\infty)\cdot(\pm\infty)\defeq+\infty,\quad
        (\pm\infty)\cdot(\mp\infty)\defeq-\infty
        % &a/(\pm\infty)\defeq0,\\
        % &(\pm\infty)/a\defeq\begin{cases}
        %     \pm\infty&\text{($0<a<+\infty$ のとき)}\\
        %     \mp\infty&\text{($-\infty<a<0$ のとき)}
        % \end{cases}
    \end{align*}
    (すべて複号同順) と定める.
    これら以外の演算は定義しない\footnote{測度論では $0\cdot(\pm\infty)=(\pm\infty)\cdot0\defeq0$ と定めることがあるが,\cite[p.12]{It63} に倣ってこの約束は用いずに進める.}.

    $\Ocal$ を $\Rbb$ の通常の位相とする.
    $\Rbb\cup\{\pm\infty\}$ を,位相
    \begin{align}
        \widetilde\Ocal\defeq\Ocal\cup\{O\cup(a,+\infty]\mid O\in\Ocal, a\in\Rbb\}\cup\{O\cup[-\infty,a)\mid O\in\Ocal, a\in\Rbb\}\cup\{\Rbb\cup\{\pm\infty\}\}
        \label{eqn:extended_real_topology}
    \end{align}
    を備えた位相空間と見なす.
    % \begin{align*}
    %     \Ocal\cup\{(a,+\infty]\mid a\in\Rbb\}\cup\{[-\infty,a)\mid a\in\Rbb\}
    % \end{align*}
    % によって生成される位相 $\widetilde\Ocal$ を入れる.
    このとき,$\Rbb\cup\{\pm\infty\}$ は閉区間 $[-1,1]$ と同相になる.
    また,$\Ocal=\widetilde\Ocal_\Rbb$ が成り立つ.
    すなわち,$\Rbb\cup\{\pm\infty\}$ の部分位相空間としての $\Rbb$ は $(\Rbb,\Ocal)$ と等しい.

    例 \ref{ex:Borel_algebra} によって $\Rbb\cup\{\pm\infty\}$ は自然に可測空間になる.
    $\Rbb\cup\{\pm\infty\}$ 上の Borel 集合族は
    \begin{align*}
        \Bfrak(\Rbb\cup\{\pm\infty\})
        =\{B\cup S\mid B\in\Bfrak(\Rbb),\ S\subset\{+\infty,-\infty\}\}
    \end{align*}
    と特徴づけられる.
    !!!TODO!!!
\end{example}

\begin{definition}
    $(X,\Afrak)$ を可測空間とする.
    次の $2$ つの条件を満たす $\mu:\Afrak\to\Rbb\cup\{+\infty\}$ を $X$ 上の測度 (measure) という.
    \begin{enumerate}[label=\textsf{(M\arabic*)},align=left]
        \item\label{item:m1} 任意の $A\in\Afrak$ に対して $\mu(A)\ge0$.また,$\mu(\emptyset)=0$.\qquad (非負性)
        \item\label{item:m2} 互いに素\footnote{pairwise disjoint.すなわち,$i\ne j$ ならば $A_i\cap A_j=\emptyset$.}な任意の $(A_i)_{i=1}^\infty\subset\Afrak$ に対して
            $\displaystyle\mu\biggl(\bigcup_{i=1}^\infty A_i\biggr)=\sum_{i=1}^\infty\mu(A_i)$.\qquad ($\sigma$-加法性)
    \end{enumerate}
    三つ組 $(X,\Afrak,\mu)$ を測度空間 (measure space) という.
\end{definition}

\begin{remark}
    測度空間 $(X,\Afrak,\mu)$ に対して次が成り立つ.
    \begin{enumerate}
        \item 任意の $A,B\in\Afrak,\ A\subset B,$ に対して $\mu(A)\le\mu(B)$.\qquad(単調性)
        \item 任意の $(A_i)_{i=1}^\infty\subset\Afrak$ に対して $\displaystyle\mu\biggl(\bigcup_{i=1}^\infty A_i\biggr)\le\sum_{i=1}^\infty\mu(A_i)$.\qquad(劣加法性)
    \end{enumerate}
\end{remark}

\begin{example}\label{ex:restriction_of_measure}
    $(X,\Afrak,\mu)$ を測度空間,$E\in\Afrak$ とする.
    このとき,例 \ref{ex:relative_sigma_algebra} の相対 $\sigma$-加法族 $\Afrak_E$ は
    $\Afrak$ の部分集合になるので,$\mu$ を $\Afrak_E$ に制限することができる.
    このようにして,$E$ は自然に測度空間 $(E,\Afrak_E,\mu\vert_{\Afrak_E})$ となる.
\end{example}

測度を具体的に構成するためには,次に示す外測度の概念が有用である.

\begin{definition}
    $X$ を集合とする.
    次の $3$ つの条件を満たす $\Gamma:2^X\to\Rbb\cup\{+\infty\}$ を $X$ 上の外測度 (outer measure) という.
    \begin{enumerate}[label=\textsf{(OM\arabic*)},align=left]
        \item\label{item:om1} 任意の $A\subset X$ に対して $\Gamma(A)\ge0$.また,$\Gamma(\emptyset)=0$.
        \item\label{item:om2} 任意の $A,B\subset X,\ A\subset B,$ に対して $\Gamma(A)\le\Gamma(B)$.
        \item\label{item:om3} 任意の $(A_i)_{i=1}^\infty\subset\Afrak$ に対して $\displaystyle\Gamma\biggl(\bigcup_{i=1}^\infty A_i\biggr)\le\sum_{i=1}^\infty\Gamma(A_i)$.
    \end{enumerate}
\end{definition}

\begin{example}\label{ex:Lebesgue_outer_measure}
    $I=(a_1,b_1]\times\cdots\times(a_n,b_n],\ -\infty\le a_i<b_i\le +\infty,$ と表される集合を $\Rbb^n$ の区間という.
    ただし,$b_i=+\infty$ のときは $(a_i,b_i]\defeq(a_i,+\infty)$ と約束する
    \footnote{これは単に区間の表記を簡単にするための規約であって,区間の表記以外には適用しない.}.
    便宜上,空集合も区間の一つと考える.
    また,区間の $n$ 次元体積を
    \begin{align*}
        &\mathrm{vol}(I)\defeq(b_1-a_1)\cdots(b_n-a_n)\in\Rbb\cup\{+\infty\},\\
        &\mathrm{vol}(\emptyset)\defeq0
    \end{align*}
    と置く.
    $\mu_\Lrm^{n,\ast}:2^{\Rbb^n}\to\Rbb\cup\{+\infty\}$ を任意の $A\subset\Rbb^n$ に対して
    \begin{align}
        \mu_\Lrm^{n,\ast}(A)\defeq\inf\left\{
            \sum_{i=1}^\infty\mathrm{vol}(I_i)
            \,\middle\vert\,
            \text{$I_1,I_2,\ldots$ は $\Rbb^n$ の区間,$A\subset\bigcup_{i=1}^\infty I_i$}
        \right\}
        \label{eqn:Lebesgue_outer_measure}
    \end{align}
    と定めると,これは $\Rbb^n$ 上の外測度になる \cite[p.25]{It63}.
    $\mu_\Lrm^{n,\ast}$ を Lebesgue 外測度という.
    \nomenclature{$\mu_\Lrm^{n,\ast}$}{$n$ 次元 Lebesgue 外測度}
\end{example}

\begin{definition}
    $\Gamma$ を集合 $X$ 上の外測度,$E\subset X$ とする.
    $E$ が $\Gamma$-可測 ($\Gamma$-measurable) であるとは,Carath\'eodory の条件:任意の $A\subset X$ に対して
    \begin{align}
        \Gamma(A)=\Gamma(A\cap E)+\Gamma(A\cap E^\crm),
        \label{eqn:Caratheodory}
    \end{align}
    が成り立つことをいう.
    $\Gamma$-可測集合の全体を $\Mfrak_\Gamma$ とおく.
    \nomenclature{$\Mfrak_\Gamma$}{$\Gamma$-可測集合の全体}
\end{definition}

\begin{remark}
    \ref{item:om3} によって,\eqref{eqn:Caratheodory} の $(\text{左辺})\le(\text{右辺})$ はいつも成り立つ.
\end{remark}

\begin{theorem}\label{thm:outer_measure_to_measure}
    $\Gamma$ を集合 $X$ 上の外測度とする.
    このとき,$\Mfrak_\Gamma$ は $X$ 上の $\sigma$-加法族で
    \begin{align*}
        \Gamma\vert_{\Mfrak_\Gamma}:\Mfrak_\Gamma\to\Rbb\cup\{+\infty\}
    \end{align*}
    は $X$ 上の測度になる.
\end{theorem}

\begin{remark}
    定理 \ref{thm:outer_measure_to_measure} によると,与えられた外測度を Carath\'eodory の条件が満たされる集合に制限することで測度が得られる.
    よって,この条件は``大きさを測れるべき集合''が満たすべき自然なものになっていると考えられる.
    しかし,この条件は一見してその意味するところが掴みにくい (と思う).
    そこで,$\Gamma=\mu_\Lrm^{n,\ast}$ の場合にこの条件が何を意味しているかを調べる.

    $E$ を $\Rbb^n$ の有界集合,$A\supset E$ を有界な区間とする.
    \eqref{eqn:Caratheodory} を次のように書き換える:
    \begin{align*}
        % \mu_\Lrm^{n,\ast}(A)=\mu_\Lrm^{n,\ast}(A\cap E)+\mu_\Lrm^{n,\ast}(A\cap E^\crm)
        % &\iff\mathrm{vol}(A)=\mu_\Lrm^{n,\ast}(E)+\mu_\Lrm^{n,\ast}(A\setminus E)\\
        % &\iff\mu_\Lrm^{n,\ast}(E)=\mathrm{vol}(A)-\mu_\Lrm^{n,\ast}(A\setminus E).
        \mu_\Lrm^{n,\ast}(E)=\mathrm{vol}(A)-\mu_\Lrm^{n,\ast}(A\setminus E).
    \end{align*}
    この右辺は,$A$ 全体の測度から $E$ の補集合の外測度を取り去ることで $E$ の大きさを測っていることに相当している ($A$ は有界なので $\infty-\infty$ の形は現れないことに注意).
    すなわち,ある意味で $E$ を内側から近似したような量になっている.
    これは Lebesgue 内測度
    \begin{align*}
        \mu_{\Lrm,\ast}^n(E)\defeq\mathrm{vol}(A)-\mu_\Lrm^{n,\ast}(A\setminus E)
    \end{align*}
    と呼ばれており,$A$ の選び方によらずに定まる.
    また,
    \begin{align}
        \mu_{\Lrm,\ast}^n(E)=\sup\left\{\mu_\Lrm^{n,\ast}(K)\mid\text{$K\subset E$,$K$ はコンパクト}\right\}
        \label{eqn:Lebesgue_inner_measure}
    \end{align}
    \nomenclature{$\mu_{\Lrm,\ast}^n$}{$n$ 次元 Lebesgue 内測度}
    という特徴づけが知られている.
    $E$ が有界とは限らない場合には \eqref{eqn:Lebesgue_inner_measure} を定義とする.
    一般に,Carath\'eodory の条件は
    \begin{align*}
        \mu_\Lrm^{n,\ast}(E)=\mu_{\Lrm,\ast}^n(E)
    \end{align*}
    と同値である (Riemann 積分論における Jordan 可測性と類似).

    Jordan 測度の場合と異なり,Lebesgue 外測度と内測度の定義の間に非対称性があることに注意する.
    たとえば,定義式が (ある程度) 対称な形になるように Lebesgue 内測度を
    \begin{align*}
        \mu_{\Lrm,\ast}^n(E)\stackrel{?}{\defeq}\sup\left\{
            \sum_{i=1}^\infty\mathrm{vol}(I_i)
            \,\middle\vert\,
            \text{$I_1,I_2,\ldots$ は互いに素な $\Rbb^n$ の区間,$E\supset\bigcup_{i=1}^\infty I_i$}
        \right\}
    \end{align*}
    と定めてもうまくいかない.
    このことは,$E=[0,1]\setminus\Qbb$ の場合を考えると了解できる
    ($\mu_{\Lrm,\ast}^n(E)=1$ になってほしいが,$[0,1]\setminus\Qbb$ に含まれるように区間を選ぶことはできないので $\mu_{\Lrm,\ast}^n(E)=0$ になってしまう).
    測度論の非対称性については \cite{ms308856} にコメントが載っている.

    % \begin{align*}
    %     \mathrm{vol}(A)-\mu_\Lrm^{n,\ast}(A\setminus E)
    %     &=\mathrm{vol}(A)-\inf\left\{
    %         \sum_{i=1}^\infty\mathrm{vol}(I_i)
    %         \,\middle\vert\,
    %         \text{$I_i$ : $\Rbb^n$ の区間,$A\setminus E\subset\bigcup_{i=1}^\infty I_i$}
    %     \right\}\\
    %     &=\mathrm{vol}(A)-\inf\left\{
    %         \sum_{i=1}^\infty\mathrm{vol}(I_i)
    %         \,\middle\vert\,
    %         \text{$I_i$ : $\Rbb^n$ の互いに素な区間,$A\setminus E\subset\bigcup_{i=1}^\infty I_i\subset A$}
    %     \right\}\\
    %     &=\sup\left\{
    %         \mathrm{vol}(A)-\sum_{i=1}^\infty\mathrm{vol}(I_i)
    %         \,\middle\vert\,
    %         \text{$I_i$ : $\Rbb^n$ の互いに素な区間,$A\setminus E\subset\bigcup_{i=1}^\infty I_i\subset A$}
    %     \right\}\\
    %     &=\sup\left\{
    %         \sum_{j=1}^\infty\mathrm{vol}(J_j)
    %         \,\middle\vert\,
    %         \text{$J_j$ : $\Rbb^n$ の互いに素な区間,$\bigcup_{j=1}^\infty J_j\subset E$}
    %     \right\}.
    % \end{align*}
    % 最後の等号では,$A$ を互いに素な区間 $I_1,I_2,\ldots,J_1,J_2,\ldots$ に分割したときに
    % \begin{itemize}
    %     \item $\displaystyle\mathrm{vol}(A)=\sum_{i=1}^\infty\mathrm{vol}(I_i)+\sum_{j=1}^\infty\mathrm{vol}(J_j)$
    %     \item $\displaystyle A\setminus E\subset\bigcup_{i=1}^\infty I_i\subset A\stackrel{\text{$A\setminus\cdot$ をとる}}{\iff}\bigcup_{j=1}^\infty J_j\subset E$
    % \end{itemize}
    % となることを用いた.
    % !!! 間違い !!! このような $J_j$ は取れないと思う
\end{remark}

\begin{proof}[定理 \ref{thm:outer_measure_to_measure} の証明]
    \ref{item:sa1}, \ref{item:sa2}, \ref{item:m1} は明らか.
    \ref{item:sa3}, \ref{item:m2} を示す.
    互いに素な $(E_i)_{i=1}^\infty\subset\Mfrak_\Gamma$ を任意にとり,$S=\bigcup_{i=1}^\infty E_i$ と置く.

    \begin{claim}\label{claim:out1}
        任意の $A\subset X$ と $k\in\Zbb_{>0}$ に対して
        \begin{align}
            \Gamma(A)\ge\sum_{i=1}^k\Gamma(A\cap E_i)+\Gamma(A\cup S^\crm).
            \label{eqn:outer_measure_claim1}
        \end{align}
    \end{claim}

    \begin{proof}[主張 \ref{claim:out1} の証明]
        $k$ について帰納法で示す.
        $k=1$ のとき,
        \begin{align*}
            \Gamma(A)
            &=\Gamma(A\cap E_1)+\Gamma(A\cap E_1^\crm)&&\because E_1\in\Mfrak_\Gamma\\
            &\ge\Gamma(A\cap E_1)+\Gamma(A\cap S^\crm)&&\because\text{\ref{item:om2}}
        \end{align*}
        となるので成立.
        $k$ で成立を仮定する.
        \eqref{eqn:outer_measure_claim1} の $A$ として $A\cap E_{k+1}^\crm$ を選ぶと
        \begin{align*}
            \Gamma(A\cap E_{k+1}^\crm)
            &\ge\sum_{i=1}^k\Gamma(A\cap E_{k+1}^\crm\cap E_i)+\Gamma(A\cap E_{k+1}^\crm\cap S^\crm)\\
            &=\sum_{i=1}^k\Gamma(A\cap E_i)+\Gamma(A\cap S^\crm).
        \end{align*}
        ゆえに,
        \begin{align*}
            \Gamma(A)
            &=\Gamma(A\cap E_{k+1})+\Gamma(A\cap E_{k+1}^\crm)&&\because E_{k+1}\in\Mfrak_\Gamma\\
            &\ge\Gamma(A\cap E_{k+1})+\sum_{i=1}^k\Gamma(A\cap E_i)+\Gamma(A\cap S^\crm)\\
            &=\sum_{i=1}^{k+1}\Gamma(A\cap E_i)+\Gamma(A\cap S^\crm)
        \end{align*}
        となって $k+1$ でも成立.
    \end{proof}

    よって,任意の $A\subset X$ に対して
    \begin{align*}
        \Gamma(A)
        &\ge\sum_{i=1}^\infty\Gamma(A\cap E_i)+\Gamma(A\cap S^\crm)&&\because\text{主張 \ref{claim:out1} で $k\to+\infty$}\\
        &\ge\Gamma\biggl(\bigcup_{i=1}^\infty(A\cap E_i)\biggr)+\Gamma(A\cap S^\crm)&&\because\text{\ref{item:om3}}\\
        &=\Gamma(A\cap S)+\Gamma(A\cap S^\crm)\\
        &\ge\Gamma(A).&&\because\text{\ref{item:om3}}
    \end{align*}
    ゆえに,
    \begin{align*}
        \Gamma(A)
        =\sum_{i=1}^\infty\Gamma(A\cap E_i)+\Gamma(A\cap S^\crm)
        =\Gamma(A\cap S)+\Gamma(A\cap S^\crm)
    \end{align*}
    となり $S\in\Mfrak_\Gamma$ が分かる.
    $A=S$ として $\Gamma(S)=\sum_{i=1}^\infty\Gamma(E_i)$ もいえるので,\ref{item:m2} が成立.

    \begin{claim}\label{claim:out2}
        任意の $E,F\in\Mfrak_\Gamma$ に対して $E\cap F,E\setminus F\in\Mfrak_\Gamma$.
    \end{claim}
    \begin{proof}[主張 \ref{claim:out2} の証明]
        \begin{align*}
            &\Gamma(A\cap(E\cap F))+\Gamma(A\cap(E\cap F)^\crm)\\
            &=\Gamma(A\cap E\cap F)+\Gamma(A\cap(E\cap F)^\crm\cap E)+\Gamma(A\cap(E\cap F)^\crm\cap E^\crm)&&\because E\in\Mfrak_\Gamma\\
            &=\Gamma(A\cap E\cap F)+\Gamma(A\cap E\cap F^\crm)+\Gamma(A\cap E^\crm)\\
            &=\Gamma(A\cap E)+\Gamma(A\cap E^\crm)&&\because F\in\Mfrak_\Gamma\\
            &=\Gamma(A)&&\because E\in\Mfrak_\Gamma
        \end{align*}
        となるので $E\cap F\in\Mfrak_\Gamma$.
        また,$E\setminus F=E\cap F^\crm\in\Mfrak_\Gamma$.
    \end{proof}

    互いに素とは限らない $(E_i)_{i=1}^\infty\subset\Mfrak_\Gamma$ に対して,主張 \ref{claim:out2} より
    \begin{align*}
        \bigcup_{i=1}^\infty E_i=E_1\cup(E_2\setminus E_1)\cup(E_3\setminus(E_1\cup E_2))\cup\cdots\in\Mfrak_\Gamma
    \end{align*}
    となって,\ref{item:sa3} が成立.
\end{proof}

\begin{example}
    $X=\Rbb^n,\Gamma=\mu_\Lrm^{n,\ast}$ (例 \ref{ex:Lebesgue_outer_measure} を参照) に対して,
    定理 $\ref{thm:outer_measure_to_measure}$ の方法で定まる $\sigma$-加法族を $\Bfrak_\Lrm^n$,
    測度を $\mu_\Lrm^n:\Bfrak_\Lrm^n\to\Rbb\cup\{+\infty\}$ と書く.
    $\Bfrak_\Lrm^n$ の元を Lebesgue 可測集合,$\mu_\Lrm^n$ を $n$ 次元 Lebesgue 測度という.
    \nomenclature{$\Bfrak_\Lrm^n$}{Lebesgue 可測集合の全体}
    \nomenclature{$\mu_\Lrm^n$}{$n$ 次元 Lebesgue 測度}
\end{example}

以上で Lebesgue 測度が構成できた.
ここで用いた構成法は次のように一般化される
\footnote{
    Lebesgue 測度の構成は,定理 \ref{thm:Hopf_extension} で $\Ffrak$ として有限個の区間の和集合 (区間塊という) の全体を,
    $m$ として区間塊の $n$ 次元体積を選ぶことに相当している.
    ただし,定理を素朴に適用して得られるのは $\sigma[\Ffrak]$
    (これは Borel 集合族 $\Bfrak^n$ と一致する.
    \cite[定理 6.4]{It63} または命題 \ref{prop:measurable_function_characterization} の証明を参照.)
    を定義域とする測度であることに注意する.
    特に,後に注意 \ref{rem:Borel_Lebesgue} で見るように $\Bfrak^n\subsetneq\Bfrak_\Lrm^n$ となるので,
    定理によって得られるのは厳密には Lebesgue 測度ではない.
    Lebesgue 測度を得るには,定理の証明において $\Gamma$ を $\sigma[\Ffrak]$ 上に制限するステップを省略すればよい.
}.

\begin{theorem}[E.\ Hopf の拡張定理,または Carath\'eodory の拡張定理]\label{thm:Hopf_extension}
    \leavevmode\par
    $X$ を集合とする.
    $\Ffrak\subset2^X$ は \ref{item:sa1}, \ref{item:sa2} および
    \begin{enumerate}[align=left]
        \item[$\textsf{(SA3)}_\textsf{fin}$] 任意の $A,B\in\Ffrak$ に対して $A\cup B\in\Ffrak$
    \end{enumerate}
    を満たすとする (このような $\Ffrak$ を有限加法族という).
    $m:\Ffrak\to\Rbb\cup\{+\infty\}$ は \ref{item:m1} および
    \begin{enumerate}[align=left]
        \item[$\textsf{(M2)}_\textsf{fin}$] 互いに素な任意の $A,B\in\Ffrak$ に対して $m(A\cup B)=m(A)+m(B)$
    \end{enumerate}
    を満たすとする (このような $m$ を有限加法的測度という).

    このとき,測度 $\mu:\sigma[\Ffrak]\to\Rbb\cup\{+\infty\}$ で $\mu\vert_\Ffrak=m$ を満たすものが存在するためには,
    $m$ が $\Ffrak$ 上で $\sigma$-加法的であること:
    \begin{itemize}
        \item 互いに素な任意の $(A_i)_{i=1}^\infty\subset\Ffrak$ で
            $\displaystyle\bigcup_{i=1}^\infty A_i\in\Ffrak$ なるものに対して
            $\displaystyle m\biggl(\bigcup_{i=1}^\infty A_i\biggr)=\sum_{i=1}^\infty m(A_i)$,
    \end{itemize}
    が必要十分である.
    さらに,ある $(X_i)_{i=1}^\infty\subset\Ffrak$ で
    $m(X_i)<\infty$ ($i=1,2,\ldots$) かつ $\displaystyle\bigcup_{i=1}^\infty X_i=X$
    なるものが存在するなら,このような $\mu$ は一意である.
\end{theorem}

\begin{proof}
    必要性は明らか.
    十分性を示す.
    まず,(\eqref{eqn:Lebesgue_outer_measure} に対応して)
    $\Gamma:2^X\to\Rbb\cup\{+\infty\}$ を任意の $A\subset X$ に対して
    \begin{align*}
        \Gamma(A)=\inf\left\{
            \sum_{i=1}^\infty m(E_i)
            \,\middle\vert\,
            E_1,E_2,\ldots\in\Ffrak,\ A\subset\bigcup_{i=1}^\infty E_i
        \right\}
    \end{align*}
    と定めると,これは $X$ 上の外測度になる.
    $\Gamma$ が $m$ の拡張になっていることを示そう.
    $E\in\Ffrak$ とする.
    $\Ffrak$ の元による $E$ の被覆として $\{E\}$ を選ぶことで $\Gamma(E)\le m(E)$ が分かる.
    一方,$(E_i)_{i=1}^\infty\subset\Ffrak$ で $E\subset\bigcup_{i=1}^\infty E_i$ なるものに対して
    \begin{align*}
        m(E)
        &=m(E_1\cap E)+m((E_2\setminus E_1)\cap E)+m((E_3\setminus(E_1\cup E_2))\cap E)+\cdots&&\because\text{$m$ は $\Ffrak$ 上 $\sigma$-加法的}\\
        &\le m(E_1)+m(E_2)+m(E_3)+\cdots&&\because\text{\ref{item:om2}}
    \end{align*}
    となるので,右辺の下限をとると $m(E)\le\Gamma(E)$ を得る.
    従って,$\Gamma$ は $m$ の拡張になっている.

    この $\Gamma$ に対して定理 \ref{thm:outer_measure_to_measure} を適用して測度空間 $(X,\Mfrak_\Gamma,\Gamma\vert_{\Mfrak_\Gamma})$ を得る.
    $\Ffrak\subset\Mfrak_\Gamma$ \cite[定理 5.2]{It63} で $\Mfrak_\Gamma$ は $\sigma$-加法族なので,
    特に $\sigma[\Ffrak]\subset\Mfrak_\Gamma$ であって,$\Gamma$ を $\sigma[\Ffrak]$ 上に制限することで所望の測度 $\mu$ が得られる.
    証明の残りの部分については \cite[定理 9.1]{It63} を参照.
\end{proof}

次に,測度空間の完備性について述べる.

\begin{definition}
    $(X,\Afrak,\mu)$ を測度空間とする.
    \begin{enumerate}
        \item $N\in\Afrak$ が零集合 (null set) であるとは,$\mu(N)=0$ となることをいう.
        \item $(X,\Afrak,\mu)$ が完備 (complete) であるとは,零集合の任意の部分集合が $\Afrak$ に属する (したがって零集合になる) ことをいう.
    \end{enumerate}
\end{definition}

\begin{proposition}
    定理 \ref{thm:outer_measure_to_measure} の測度空間 $(X,\Mfrak_\Gamma,\Gamma\vert_{\Mfrak_\Gamma})$ は完備.
    特に,$(\Rbb^n,\Bfrak_\Lrm^n,\mu_\Lrm^n)$ は完備.
\end{proposition}

\begin{proof}
    零集合 $N\in\Afrak$ と任意の $N'\subset N$ に対して,$\Gamma(N')\le\Gamma(N)=0$ なので $\Gamma(N')=0$.
    よって,任意の $A\subset X$ に対して
    \begin{align*}
        \Gamma(A)
        \le\Gamma(A\cap N')+\Gamma(A\cap N'^\crm)
        =\Gamma(A\cap N'^\crm)
        \le\Gamma(A)
    \end{align*}
    となり $N'\in\Mfrak_\Gamma$ を得る.
\end{proof}

\begin{theorem}\label{thm:measure_completion}
    測度空間 $(X,\Afrak,\mu)$ に対して,$\mu^\ast:2^X\to\Rbb\cup\{+\infty\}$ を
    \begin{align*}
        \mu^\ast(A)\defeq\inf\left\{
            \mu(B)\,\middle\vert\,A\subset B\in\Afrak
        \right\}
    \end{align*}
    と定めると,これは $X$ 上の外測度になる.
    さらに,$\mu^\ast$ に対して定理 \ref{thm:outer_measure_to_measure} を適用して得られる完備測度空間
    $(X,\Mfrak_{\mu^*},\mu^\ast\vert_{\Mfrak_{\mu^\ast}})$ は
    \begin{align*}
        \Afrak\subset\Mfrak_{\mu^*},\quad
        \mu^\ast\vert_\Afrak=\mu
    \end{align*}
    を満たす.
\end{theorem}

\begin{proof}
    \cite[定理 8.4]{It63} を参照.
\end{proof}

\begin{definition}
    定理 \ref{thm:measure_completion} の測度空間 $(X,\Mfrak_{\mu^*},\mu^\ast\vert_{\Mfrak_{\mu^\ast}})$ を
    $(X,\Afrak,\mu)$ の完備化 (completion) という.
\end{definition}

常に完備化した空間で考えた方がいいというわけではない! % TODO
可測関数の議論(次回)を参照

\begin{remark}\label{rem:Borel_Lebesgue}
    \leavevmode
    \begin{enumerate}
        \item
            $\Rbb^n$ の Borel 集合は Lebesgue 可測である.
            すなわち $\Bfrak^n\subset\Bfrak_\Lrm^n$.
            また,測度空間 $(\Rbb^n,\Bfrak^n,\mu_\Lrm^n\vert_{\Bfrak^n})$ の完備化は
            $(\Rbb^n,\Bfrak_\Lrm^n,\mu_\Lrm^n)$ と一致する.
            前者については \cite[定理 7.2]{It63} を,後者については \cite[p.49]{It63} を参照.

        \item
            $\Bfrak^n\subsetneq\Bfrak_\Lrm^n\subsetneq2^{\Rbb^n}$ である.
            より正確には,$\cfrak$ を連続体濃度として
            \begin{align}
                |\Bfrak^n|=\cfrak,\quad
                |\Bfrak_\Lrm^n|=2^\cfrak,\quad
                |2^{\Rbb^n}|=2^\cfrak,\quad
                |2^{\Rbb^n}\setminus\Bfrak_\Lrm^n|=2^\cfrak
            \end{align}
            が成り立つ.

            一つ目の等号は超限帰納法を用いて Borel 集合を具体的に``生成''することによって示される\cite{ms70880}.
            二つ目の等号は次のようにして示される:
            $N\in\Bfrak_\Lrm^n$ で $\mu_\Lrm^n(N)=0$ かつ $|N|=\cfrak$ となるもの (たとえば Cantor 集合) を一つとる.
            Lebesgue 測度の完備性から $N$ の任意の部分集合は Lebesgue 可測なので結論が従う.
            三つ目の等号は $|\Rbb^n|=\cfrak$ から従う.
            四つ目の等号は,$A\in2^{\Rbb^n}\setminus\Bfrak_\Lrm^n$ を固定したとき,
            任意の $X\in\Bfrak_\Lrm^n$ に対して $A\ominus X\in\Bfrak_\Lrm^n$ となることから従う ($\ominus$ は集合の対称差).

        \item Lebesgue 非可測集合の例
            \begin{enumerate}
                \item
                    区間 $G=[0,1)$ は $\mathrm{mod}\ 1$ での和について Abel 群をなす.
                    $G$ を部分群 $H=\Qbb\cap[0,1)$ で割った剰余群 $G/H$ について,その完全代表系 $A$ を一つとり固定する.
                    このとき,$G=\bigcup_{r\in H}(r+A)$ と書けて,しかも右辺は非交和である.
                    $A$ が Lebesgue 可測であると仮定すると,$H$ は可算集合で Lebesgue 測度は平行移動不変なので
                    \begin{align*}
                        1
                        =\mu_\Lrm^1(G)
                        =\sum_{r\in H}\mu_\Lrm^1(r+A)
                        =\sum_{r\in H}\mu_\Lrm^1(A)
                    \end{align*}
                    となる.
                    $\mu_\Lrm^1(A)=0,>0$ のいずれであっても矛盾が起きるので,$A$ は Lebesgue 非可測である.
                    $A$ を Vitali 集合という.

                \item
                    $\Rbb^3$ における半径 $1$ の球体を有限個の集合 $A_1,\ldots,A_m$ に分割して,
                    それらを回転と平行移動によって組み替えることで,半径 $1$ の球体を二つ作ることができる.
                    これは Banach--Tarski のパラドックスと呼ばれている.
                    YouTube \cite{yts86-Z-CbaHA} に綺麗なグラフィックスを用いた解説がある (英語だが字幕がついている).

                    このとき,$A_1,\ldots,A_m$ の少なくとも一つは Lebesgue 可測ではない.
                    実際,これらがすべて Lebesgue 可測であると仮定すると,Lebesgue 測度が回転と平行移動について不変であることから
                    $\mu_\Lrm^3(\text{球体})=2\times\mu_\Lrm^3(\text{球体})$ がいえてしまい矛盾が起きる.
            \end{enumerate}

        \item Borel でない Lebesgue 可測集合の例\cite{ms253786} (Lusin).
            $x\in\Rbb\setminus\Qbb$ の連分数展開を
            \begin{align*}
                x=a_0+\frac{1}{a_1+\dfrac{1}{a_2+\cdots}}
            \end{align*}
            と書くと,
            \begin{align*}
                \left\{
                    x\in\Rbb\setminus\Qbb\,\middle\vert\,
                    \begin{array}{c}
                        \text{ある $0\le i_0<i_1<\cdots$ に対して}\\
                        a_{i_0}\mid a_{i_1}\mid a_{i_2}\mid\cdots
                    \end{array}
                \right\}\in\Bfrak_\Lrm^1\setminus\Bfrak^1.
            \end{align*}
    \end{enumerate}
\end{remark}
