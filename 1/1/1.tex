\subsubsection{可測空間}\label{sec:measurable_space}

\begin{definition}
    $X$ を集合とする.
    次の三つの条件を満たす $\Afrak\subset2^X$ を $X$ 上の $\sigma$-加法族 ($\sigma$-algebra) という.
    \begin{enumerate}[label=\textsf{(SA\arabic*)},align=left]
        \item\label{item:sa1} $\emptyset\in\Afrak$.
        \item\label{item:sa2} 任意の $A\in\Afrak$ に対して $A^\crm\in\Afrak$.\quad (ここで,$A^\crm\defeq X\setminus A$)
        \item\label{item:sa3} 任意の $(A_n)_{n=1}^\infty\subset\Afrak$ に対して $\displaystyle\bigcup_{n=1}^\infty A_n\in\Afrak$.
    \end{enumerate}
    \nomenclature{$A^\crm$}{集合 $A$ の補集合}
    組 $(X,\Afrak)$ を可測空間 (measurable space),$\Afrak$ の元を可測集合 (measurable set) という.
\end{definition}

\begin{remark}\label{rem:measurable_space_other_axioms}
    可測空間 $(X,\Afrak)$ に対して次が成り立つ.
    \begin{enumerate}
        \item $X\in\Afrak$.
        \item 任意の $(A_n)_{n=1}^\infty\subset\Afrak$ に対して $\displaystyle\bigcap_{n=1}^\infty A_n\in\Afrak$.
    \end{enumerate}
    \textrm{i)} は \ref{item:sa1} と \ref{item:sa2} から,\textrm{ii)} は \ref{item:sa2} と \ref{item:sa3} から従う.
\end{remark}

\begin{example}
    $\Afrak=\{\emptyset,X\},2^X$ はいずれも $X$ 上の $\sigma$-加法族になる.
\end{example}

\begin{example}\label{ex:relative_sigma_algebra}
    可測空間 $(X,\Afrak)$ と $E\subset X$ に対して,
    \begin{align*}
        \Afrak_E\defeq\{A\cap E\mid A\in\Afrak\}
    \end{align*}
    は $E$ 上の $\sigma$-加法族であって,$E$ は自然に可測空間 $(E,\Afrak_E)$ となる.
    $\Afrak_E$ を $\Afrak$ から誘導される $E$ 上の相対 $\sigma$-加法族という.
    \nomenclature{$\Afrak_E$}{$\Afrak$ から誘導される $E$ 上の相対 $\sigma$-加法族}
    ここでは $E\in\Afrak$ を仮定する必要がないことに注意 (例 \ref{ex:restriction_of_measure} も参照).
\end{example}

\begin{example}
    集合 $X$ 上の $\sigma$-加法族の族 $\{\Afrak_\lambda\}_{\lambda\in\Lambda}$ に対して,
        $\bigcap_{\lambda\in\Lambda}\Afrak_\lambda$
    も $X$ 上の $\sigma$-加法族になる.

    $S\subset 2^X$に対して,$S$ を含む (包含関係について) 最小の $\sigma$-加法族が
    \begin{align*}
        \sigma(S)\defeq\bigcap_{\substack{\text{$\Afrak$: $\sigma$-algebra on $X$}\\S\subset\Afrak}}\Afrak
    \end{align*}
    として一意に定まる.
    $\sigma(S)$ を $S$ によって生成される $X$ 上の $\sigma$-加法族という.
    \nomenclature{$\sigma(S)$}{$S\subset2^X$ によって生成される $X$ 上の $\sigma$-加法族}
\end{example}

\begin{example}\label{ex:product_sigma_algebra}
    $((X_\lambda,\Afrak_\lambda))_{\lambda\in\Lambda}$ を可測空間の族とする.
    直積集合を $X=\prod_{\lambda\in\Lambda}X_\lambda$,
    $X$ から各 $X_\lambda$ への自然な射影を $\pi_\lambda:X\to X_\lambda$ と置く.
    このとき,$X$ 上の $\sigma$-加法族
    \begin{align*}
        \bigotimes_{\lambda\in\Lambda}\Afrak_\lambda
        \defeq\sigma(\{\pi_\lambda^{-1}(A)\mid A\in\Afrak_\lambda,\ \lambda\in\Lambda\})
    \end{align*}
    を $(\Afrak_\lambda)_{\lambda\in\Lambda}$ の直積 $\sigma$-加法族といい,
    可測空間 $\left(X,\bigotimes_{\lambda\in\Lambda}\Afrak_\lambda\right)$ を
    $((X_\lambda,\Afrak_\lambda))_{\lambda\in\Lambda}$ の直積可測空間という.
    \nomenclature{$\bigotimes_{\lambda\in\Lambda}\Afrak_\lambda$}{$(\Afrak_\lambda)_{\lambda\in\Lambda}$ の直積 $\sigma$-加法族}
    $\Lambda$ が有限集合 $\{1,2,\ldots,n\}$ のときは直積 $\sigma$-加法族を
    $\Afrak_1\otimes\cdots\otimes\Afrak_n$ や $\bigotimes_{k=1}^n\Afrak_k$ とも書く.
    また,$\Lambda$ が高々可算な集合のときは
    \begin{align*}
        \bigotimes_{\lambda\in\Lambda}\Afrak_\lambda
        =\sigma\biggl(\prod_{\lambda\in\Lambda}\Afrak_\lambda\biggr)
    \end{align*}
    が成り立つ\footnote{
        有限個の位相空間についてそれらの直積位相 (product topology) と箱位相 (box topology) が一致する,という事実とよく似ている.
        位相が有限個の開集合の intersection について閉じているのに対して,
        $\sigma$-加法族は可算無限個の可測集合の intersection について閉じている (注意 \ref{rem:measurable_space_other_axioms})
        ので,このようなことが成り立つ.
    }.
    ここで,右辺の $\prod_{\lambda\in\Lambda}\Afrak_\lambda$ は集合としての直積を表す.
\end{example}

\begin{example}\label{ex:Borel_algebra}
    位相空間 $(X,\Ocal)$ に対して,$X$ の開集合系によって生成される $\sigma$-加法族
    $\Bfrak(X)\defeq\sigma(\Ocal)$ を Borel 集合族 (Borel algebra),
    $\Bfrak(X)$ の元を Borel 集合 (Borel set) という.
    \nomenclature{$\Bfrak(X)$}{位相空間 $X$ 上の Borel 集合族}
    また,$\Bfrak^n\defeq\Bfrak(\Rbb^n)$ と置く.
    \nomenclature{$\Bfrak^n$}{$\Rbb^n$ 上の Borel 集合族}
    このようにして,任意の位相空間は自然に可測空間と見なすことができる.
\end{example}

\begin{example}\label{ex:extended_real}
    実数体 $\Rbb$ に形式的な元 $+\infty,-\infty$ を加えた集合 $\Rbb\cup\{\pm\infty\}$ を考える.
    $a\in\Rbb$ に対して,順序関係を $-\infty<a<+\infty$ と定める.
    $\Rbb$ における区間の記号に倣って $[-\infty,+\infty]\defeq\Rbb\cup\{\pm\infty\}$ とも書く.
    $[0,+\infty]$ なども同様に定義する.
    また,$a\in\Rbb$ に対して,演算を
    \begin{align*}
        &(\pm\infty)+a=a+(\pm\infty)\defeq\pm\infty,\quad
        (\pm\infty)+(\pm\infty)\defeq\pm\infty,\\
        &(\pm\infty)-a=a-(\mp\infty)\defeq\pm\infty,\quad
        (\pm\infty)-(\mp\infty)\defeq\pm\infty,\\
        &-(\pm\infty)\defeq\mp\infty\\
        &a\cdot(\pm\infty)=(\pm\infty)\cdot a\defeq\begin{cases}
            \pm\infty&\text{if $0<a<+\infty$}\\
            \mp\infty&\text{if $-\infty<a<0$},
        \end{cases}\\
        &(\pm\infty)\cdot(\pm\infty)\defeq+\infty,\quad
        (\pm\infty)\cdot(\mp\infty)\defeq-\infty
        % &a/(\pm\infty)\defeq0,\\
        % &(\pm\infty)/a\defeq\begin{cases}
        %     \pm\infty&\text{if $0<a<+\infty$}\\
        %     \mp\infty&\text{if $-\infty<a<0$}
        % \end{cases}
    \end{align*}
    (すべて複号同順) と定める.
    これら以外の演算は定義しない\footnote{測度論では $0\cdot(\pm\infty)=(\pm\infty)\cdot0\defeq0$ と定めることがあるが,\cite[p.12]{It63} に倣ってこの約束は用いずに進める.}.

    $\Ocal$ を $\Rbb$ の通常の位相とする.
    $[-\infty,+\infty]$ を位相
    \begin{align}
        \widetilde\Ocal
        \defeq\Ocal
        \cup\{O\cup(a,+\infty]\mid O\in\Ocal, a\in\Rbb\}
        \cup\{O\cup[-\infty,a)\mid O\in\Ocal, a\in\Rbb\}
        \cup\{[-\infty,+\infty]\}
        \label{eqn:extended_real_topology}
    \end{align}
    を備えた位相空間と見なす.
    % \begin{align*}
    %     \Ocal\cup\{(a,+\infty]\mid a\in\Rbb\}\cup\{[-\infty,a)\mid a\in\Rbb\}
    % \end{align*}
    % によって生成される位相 $\widetilde\Ocal$ を入れる.
    このとき,$[-\infty,+\infty]$ は閉区間 $[-1,1]$ と同相になる.
    また,$\{O\cap\Rbb\mid O\in\widetilde\Ocal\}=\Ocal$ が成り立つ.
    すなわち,$[-\infty,+\infty]$ の部分位相空間としての $\Rbb$ は $(\Rbb,\Ocal)$ と等しい.

    例 \ref{ex:Borel_algebra} によって $[-\infty,+\infty]$ は自然に可測空間になる.
    $[-\infty,+\infty]$ 上の Borel 集合族は
    \begin{align*}
        \Bfrak([-\infty,+\infty])
        =\{B\cup S\mid B\in\Bfrak(\Rbb),\ S\subset\{+\infty,-\infty\}\}
    \end{align*}
    と特徴づけられる.
    {\color{red} TODO もう少し補足したいかも}
\end{example}
