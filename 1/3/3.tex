\subsubsection{Fubini の定理}

{\color{red}この節は書きかけ!!!}

この節では,二つの積分の順序交換を保証する Fubini の定理について述べる.
また,直積測度空間を完備化した場合の結果についても言及する.

\begin{theorem}[Fubini]
    $(X,\Afrak_1,\mu_1),(Y,\Afrak_2,\mu_2)$ を $\sigma$-有限な測度空間とする.
    これらの直積測度空間を $(X\times Y,\Afrak_1\otimes\Afrak_2,\mu_1\otimes\mu_2)$ と置く (定義 \ref{def:product_measure}).
    \begin{enumerate}
        \item 非負値可測関数 $f:X\times Y\to\Rbb\cup\{\pm\infty\}$ に対して,次が成り立つ.
            \begin{enumerate}
                \item[1-a)] 任意の $x\in X$ に対して,$Y\ni y\mapsto f(x,y)\in\Rbb\cup\{\pm\infty\}$ は可測である.
                \item[1-b)] 任意の $y\in Y$ に対して,$X\ni x\mapsto f(x,y)\in\Rbb\cup\{\pm\infty\}$ は可測である.
                \item[2-a)] $\displaystyle X\ni x\mapsto\int_Yf(x,y)\,\drm\mu_2(y)$ は可測である.
                \item[2-b)] $\displaystyle Y\ni y\mapsto\int_Xf(x,y)\,\drm\mu_1(x)$ は可測である.
                \item[3)]
                    $\displaystyle
                        \int_{X\times Y}f(x,y)\,\drm(\mu_1\otimes\mu_2)(x,y)
                        =
                    $
            \end{enumerate}
        \item
            $f:X_1\times X_2\to\Cbb$ を可測関数とする.
            \begin{enumerate}
                \item
                    ほとんどいたる所の $x\in X$ に対して,$Y\ni y\mapsto f(x,y)\in K$ は可積分である.
                    また,ほとんどいたる所の $y\in Y$ に対して,$X\ni x\mapsto f(x,y)\in K$ は可積分である.
                \item 2
                \item 3
            \end{enumerate}
    \end{enumerate}
\end{theorem}
