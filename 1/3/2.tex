\subsubsection{収束定理}

この節では,Lebesgue 積分を応用する上で非常に重要となる,積分と極限の順序交換を保証する定理たちについてまとめる.

\begin{theorem}[単調収束定理]\label{thm:monotone_convergence}
    $(X,\Afrak,\mu)$ を測度空間とする.
    可測関数列 $(f_n:X\to[-\infty,+\infty])_{n=1}^\infty$ と可測関数 $f:X\to[-\infty,+\infty]$ は次を満たすとする.
    \begin{enumerate}
        \item $0\le f_1(x)\le f_2(x)\le\cdots\quad\text{$\mu$-a.e.\ $x\in X$}.$
        \item $\displaystyle\lim_{n\to+\infty}f_n(x)=f(x)\quad\text{$\mu$-a.e.\ $x\in X$}$.
    \end{enumerate}
    このとき,
    \begin{align}
        \lim_{n\to+\infty}\int_Xf_n\,\drm\mu=\int_Xf\,\drm\mu
        \label{eqn:mct_conclusion}
    \end{align}
    が成り立つ\footnote{
        $(f_n)_{n=1}^\infty$ の単調増加性から,\textrm{ii)} を仮定せずとも,
        $f_n$ はある可測関数に $n\to+\infty$ でほとんどいたる所各点収束することが証明される.
    }\footnote{\eqref{eqn:mct_conclusion} の両辺が共に $+\infty$ であるような状況も許容する.}.
\end{theorem}

\begin{proof}
    \cite[定理 13.2]{It63} を参照.
\end{proof}

\begin{remark}
    定理 \ref{thm:monotone_convergence} \textrm{i)} の仮定 $f_1\ge0\ \text{$\mu$-a.e.\ $X$}$ は,
    $\int_Xf_1\,\drm\mu$ が存在して
    \begin{align*}
        \int_Xf_1\,\drm\mu>-\infty
    \end{align*}
    と緩められる.
\end{remark}

\begin{lemma}[Fatou の補題]\label{lem:Fatou}
    $(X,\Afrak,\mu)$ を測度空間とする.
    可測関数列 $(f_n:X\to[-\infty,+\infty])_{n=1}^\infty$ は,
    各 $n$ に対して $f_n\ge0\ \text{$\mu$-a.e.\ $X$}$ を満たすとする.
    このとき,
    \begin{align*}
        \int_X\liminf_{n\to+\infty}f_n\,\drm\mu\le\liminf_{n\to+\infty}\int_Xf_n\,\drm\mu
    \end{align*}
    が成り立つ.
\end{lemma}

\begin{proof}
    $g_n(x)\defeq\inf_{k\ge n}f_k(x)$ と置くと,
    $(g_n)_{n=1}^\infty$ は単調増加な可測関数列であって $g_1\ge0\ \text{$\mu$-a.e.\ $X$}$ が成り立つ.
    従って,定理 \ref{thm:monotone_convergence} が適用できて
    \begin{align}
        \lim_{n\to+\infty}\int_Xg_n\,\drm\mu
        =\int_X\lim_{n\to+\infty}g_n\,\drm\mu
        =\int_X\sup_{n\ge1}g_n\,\drm\mu
        =\int_X\liminf_{n\to+\infty}f_n\,\drm\mu.\label{eqn:Fatou_mid_1}
    \end{align}
    一方,任意の $x\in X$ に対して $g_n(x)\le f_n(x)$ となるので,$\displaystyle\int_Xg_n\,\drm\mu\le\int_Xf_n\,\drm\mu$.
    ゆえに
    \begin{align}
        \liminf_{n\to+\infty}\int_Xg_n\,\drm\mu\le\liminf_{n\to+\infty}\int_Xf_n\,\drm\mu.
        \label{eqn:Fatou_mid_2}
    \end{align}
    \eqref{eqn:Fatou_mid_1}, \eqref{eqn:Fatou_mid_2} より結論を得る.
\end{proof}

\begin{theorem}[Lebesgue の優収束定理]\label{thm:dominated_convergence}
    $(X,\Afrak,\mu)$ を測度空間,$Y=[-\infty,+\infty]$ または $\Cbb$ とする.
    可測関数列 $(f_n:X\to Y)_{n=1}^\infty$ と可測関数 $f:X\to Y$ は次を満たすとする.
    \begin{enumerate}
        \item $\displaystyle\lim_{n\to+\infty}f_n(x)=f(x)\quad\text{$\mu$-a.e.\ $x\in X$}$.
        \item ある可積分関数 $\varphi:X\to[0,+\infty]$ が存在して,
            各 $n$ に対して $|f_n(x)|\le\varphi(x)\ \text{$\mu$-a.e.\ $x\in X$}$ が成り立つ.
    \end{enumerate}
    このとき,$f_n,f$ も可積分であって
    \begin{align*}
        \lim_{n\to+\infty}\int_Xf_n\,\drm\mu=\int_Xf\,\drm\mu
    \end{align*}
    が成り立つ.
\end{theorem}

\begin{proof}
    $Y=[-\infty,+\infty]$ のときを示せば十分.
    仮定 \textrm{ii)} と定理 \ref{thm:TODO (monotonicity)} より $f_n,f$ は可積分である.
    仮定 \textrm{ii)} より $\varphi\pm f_n\ge0\ \text{$\mu$-a.e.\ $X$}$ となるので,補題 \ref{lem:Fatou} が適用できて,
    \begin{align*}
        \int_X(\varphi+f)\,\drm\mu\le\liminf_{n\to+\infty}\int_X(\varphi+f_n)\,\drm\mu,\qquad
        \int_X(\varphi-f)\,\drm\mu\le\liminf_{n\to+\infty}\int_X(\varphi-f_n)\,\drm\mu
    \end{align*}
    を得る.
    これらを $\displaystyle\int_X\varphi\,\drm\mu<+\infty$ に注意して整理すると,それぞれ
    \begin{align*}
        \int_Xf\,\drm\mu\le\liminf_{n\to+\infty}\int_Xf_n\,\drm\mu,\qquad
        \int_Xf\,\drm\mu\ge\limsup_{n\to+\infty}\int_Xf_n\,\drm\mu
    \end{align*}
    となるので,結論が従う.
\end{proof}
